\chapter{Distribuições abordadas} % (fold)
\label{sec:distribui_es_abordadas}

\section{Breve descrição} % (fold)
\label{sec:breve_descri_o}

Dês do lançamento das distribuições Debian e Slackware em 1993, diversas distribuições surgiram e desapareceram. Independente da distribuição Linux hoje, provavelmente era será descendente de uma destas distribuições:

\begin{itemize}
	\item Debian
	\item Red Hat
	\item Slackware
	\item Arch
\end{itemize}

Estas distribuições representam o berço de grande parte das distribuições existentes hoje. Assim, este trabalho buscará trabalhar o mais próximo delas, com o intuito de garantir compatibilidade com o maior número possível de distribuições.

Desta forma, os gerenciadores de pacotes que serão levados em consideração neste trabalho serão:

\begin{itemize}
	\item Debian
	\begin{itemize}
		\item Apt
	\end{itemize}
	\item Fedora (\textit{Distribuição oriunda do RedHat})
	\begin{itemize}
		\item Yum
	\end{itemize}
	% \item OpenSUSE (\textit{Distribuição oriunda do Slackware})
	% \begin{itemize}
	% 	\item Zypper
	% \end{itemize}
	\item Arch
	\begin{itemize}
		\item Pacman
	\end{itemize}
\end{itemize}

\section{Comandos Básicos} % (fold)
\label{sec:comandos_b_sicos}


\subsection{Apt} % (fold)
\label{sub:apt}

Dentre os comandos que a ferramenta disponibiliza, os comandos básicos (procura por pacote, instalar pacote e remover pacote) possuem a seguinte sintaxe:

\begin{itemize}
	\item Instalar
	\item Remover
	\item Procurar pacote
\end{itemize}


\subsection{Yum} % (fold)
\label{sub:apt}

Dentre os comandos que a ferramenta disponibiliza, os comandos básicos (procura por pacote, instalar pacote e remover pacote) possuem a seguinte sintaxe:

\begin{itemize}
	\item Instalar
	\item Remover
	\item Procurar pacote
\end{itemize}
