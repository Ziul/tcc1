\section{Smith-Waterman} % (fold)
\label{sec:smith_waterman}


Proposto em 1981 por \textit{Temple F. Smith} e \textit{Michael S. Waterman}\cite{smith1981identification} com o intuito de localização de sequencias moleculares semelhantes, este algoritmo de programação dinâmica é hoje amplamente utilizado para a localização de regiões similares entre \textit{strings} e nas sequências de proteínas ou nucleotídeos.

O algoritmo trabalha com o alinhamento de \textit{strings} no intuito de  buscar a maior semelhança entre elas. Considerando este fator, é interessante que as duas \textit{strings} contenham uma quantidade de caracteres igual ou próxima, visto que  a \textit{string} com  maior quantidade de caracteres provavelmente terá suas bordas ignoradas no final. Como resultado, podemos obter melhores resultados com comparações de \textit{strings} que os algoritmos de \textit{Leveinstein} quando tratamos de busca por radicais. Todavia uma comparação usando palavras com alta distinção de quantidade de caracteres poderá resultar em uma perca de desempenho pela necessidade de permutações necessárias para localizar o melhor alinhamento.

% Explanação em http://en.wikipedia.org/wiki/Smith%E2%80%93Waterman_algorithm
\subsection{Algoritmo} % (fold)
\label{sub:algoritmo}

Considerando as \textit{strings} $a$ e $b$ de tamanhos $m,n$ respectivamente, é montada uma matriz $H$ de acordo com as seguintes regras:

\begin{align*}
	H(i,0) =& 0, 0 \leq i \leq m \\
	H(0,j) =& 0, 0 \leq j \leq n \\
	H(i,j) =& \text{max}
	\begin{Bmatrix}
		0  &\\
		H(i-1,j-1) + s(a_i,b_j) & \text{Correspondências}\\
		\underset{k\geq1}{\text{max}} \left\{H(i-k,j) + W_k\right\} & \text{Remoções}\\
		\underset{l\geq1}{\text{max}} \left\{H(i,j-l) + W_l\right\} & \text{Inserções}
	\end{Bmatrix}, 1 \leq i \leq m, 1 \leq j \leq n
\end{align*}

Sendo:
\begin{itemize}
	\item $s(a,b)$: a função de similaridade do alfabeto
	\item $W_i$: a nota de lacuna\footnote{Também conhecida como \textit{gap-scoring} ou \textit{gap-penalty}.}
\end{itemize}

Montada a matriz $H$, para obter o melhor alinhamento, o algoritmo dita começar com o maior valor na matriz $(i,j)$, vá então retornando em sentido a celula $(0,0)$, seguindo pelos pontos $(i - 1,j), (i, j - 1)$, ou $(i - 1, j - 1)$.

\subsection{Implementação do algoritmo} % (fold)
\label{sub:implementa_o_do_algoritmo}

Como comentado na \nameref{cha:metodologia}, este projeto tem como intuito desenvolver uma comparação dos algoritmos de forma apenas comparativa no momento, sem o intuito de verificar se é possível de otimizações no algoritmo. Sendo assim, estão sendo utilizadas bibliotecas já desenvolvidas e que estão  disponíveis para uso. Para o algoritmo de \textit{Smith-Waterman} não foi diferente, sendo utilizada uma das bibliotecas disponíveis para \textit{Python} chamada \textbf{swalign}\footnote{A biblioteca pode ser encontrada em \url{https://pypi.python.org/pypi/swalign/}, tendo seu código disponibilizado em \url{https://github.com/mbreese/swalign}.}

% subsection implementa_o_do_algoritmo (end)