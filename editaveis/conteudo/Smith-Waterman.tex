\subsection{Smith-Waterman} % (fold)
\label{sec:smith_waterman}


Proposto em 1981 por \textit{Temple F. Smith} e \textit{Michael S. Waterman}\cite{smith1981identification} com o intuito de localização de sequências moleculares semelhantes, este algoritmo de programação dinâmica é hoje amplamente utilizado para a localização de regiões similares entre \textit{strings} e nas sequências de proteínas ou nucleotídeos.

O algoritmo trabalha com o alinhamento de \textit{strings} no intuito de  buscar a maior semelhança entre elas. Considerando este fator, é interessante que as duas \textit{strings} contenham uma quantidade de caracteres igual ou próxima, visto que  a \textit{string} com  maior quantidade de caracteres provavelmente terá suas bordas ignoradas no final. Como resultado, podemos obter melhores resultados com comparações de \textit{strings} que os algoritmos de \textit{Leveinstein} quando tratamos de busca por radicais. Todavia uma comparação usando palavras com alta distinção de quantidade de caracteres poderá resultar em uma perca de desempenho pela necessidade de permutações necessárias para localizar o melhor alinhamento.

% Explanação em http://en.wikipedia.org/wiki/Smith%E2%80%93Waterman_algorithm
% TODO: Melhorar essa explanação de bosta
\subsubsection{Algoritmo} % (fold)
\label{sub:algoritmo}

De acordo com o artigo de Smith e Waterman \cite{smith1981identification}, o algoritmo pode ser apresentado como:

Considerando as \textit{strings} $a$ e $b$ de tamanhos $n,m$ respectivamente e $s(a,b)$ a função de similaridade do universo a ser considerado\footnote{Smith considerava apenas as letras \textit{A,C,U,G} devido ao algoritmo ser voltado para identificação de similidades genéticas, todavia a generalização para as demais letras do alfabeto não invalida o algoritmo.}, é montada uma matriz $H$  no intuito de encontrar os dois pares com maior grau de similaridade formada por $W_i$ sendo a nota de lacuna\footnote{Também conhecida como \textit{gap-scoring} ou \textit{gap-penalty}.}.

\begin{equation*}
	H_{k0} = H_{0l} = 0 \text{ para } 0\leq k \leq n \text{ e } 0\leq l \leq m.
\end{equation*}

Valores preliminares de $H$ tem a interpretação de que $H_{ij}$ tem a máxima similaridade de duas \textit{strings} que possuem final $a_i$ e $b_j$ respectivamente. Os valores de $H_ij$ podem ser obtidos através da relação:
% \begin{align*}
% 	H(i,0) =& 0, 0 \leq i \leq m \\
% 	H(0,j) =& 0, 0 \leq j \leq n \\
% 	H(i,j) =& \text{max}
% 	\begin{Bmatrix}
% 		H(i-1,j-1) + s(a_i,b_j) & \text{Correspondências}\\
% 		\underset{k\geq1}{\text{max}} \left\{H(i-k,j) - W_k\right\} & \text{Remoções}\\
% 		\underset{l\geq1}{\text{max}} \left\{H(i,j-l) - W_l\right\} & \text{Inserções}\\
% 	\end{Bmatrix},\\
% 	1 \leq i \leq n& \text{ e } 1 \leq j \leq m
% \end{align*}
\begin{eqnarray*}
	H(i,0) &=& 0, \, \, \, \, 0 \leq i \leq m \\
	H(0,j) &=& 0, \, \, \, \, 0 \leq j \leq n \\
	H(i,j) &=& \text{max}\lbrace A(i,j), B(i,j), C(i,j), 0\rbrace, \, \, \, \, 1 \leq i \leq n \text{ e } 1 \leq j \leq m
\end{eqnarray*}
onde
\begin{eqnarray*}
		A(i, j) &=& H_{i-1,j-1} + s(a_i,b_j) \\
		B(i, j) &=& \underset{k\geq1}{\text{max}} \left\{H_{i-k,j} - W_k\right\}\\ 
		C(i, j) &=& \underset{l\geq1}{\text{max}} \left\{H_{i,j-l} - W_l\right\}
\end{eqnarray*}

Montada a matriz $H$, para obter o melhor alinhamento, o algoritmo dita começar com o maior valor na matriz $(i,j)$, vá então retornando em sentido a célula $(0,0)$, seguindo pelos pontos $(i - 1,j), (i, j - 1)$, ou $(i - 1, j - 1)$.

A formula para $H_{ij}$ considera as possibilidades de fim da \textit{string} para qualquer $a_i$ ou $b_j$, tal que:

\begin{enumerate}
	\item Se $a_i$ e $b_j$ forem associados, a similaridade será dada por:
	\begin{equation*}
		H_{i-1,j-1} + s(a_i,b_j)
	\end{equation*}
	\item Se $a_i$ esta ao fim de uma remoção de largura $k$, a similaridade será dada por:
	\begin{equation*}
		H_{i-k,j} - W_k
	\end{equation*}
	\item Se $b_i$ esta ao fim de uma remoção de largura $l$, a similaridade será dada por:
	\begin{equation*}
		H_{i,j-l} - W_l
	\end{equation*}
	\item Por fim é incluído sentinelas a fim de precaver o calculo de similaridades negativas.
\end{enumerate}

\subsubsection{Implementação do algoritmo} % (fold)
\label{sub:implementa_o_do_algoritmo}

Este projeto tem como intuito desenvolver uma comparação dos algoritmos de forma apenas comparativa no momento, sem o intuito de verificar se é possível de otimizações no algoritmo. Sendo assim, estão sendo utilizadas bibliotecas já desenvolvidas e que estão  disponíveis para uso. Para o algoritmo de \textit{Smith-Waterman} não foi diferente, sendo utilizada uma das bibliotecas disponíveis para \textit{Python} chamada \textbf{swalign}\footnote{A biblioteca pode ser encontrada em \url{https://pypi.python.org/pypi/swalign/}, tendo seu código disponibilizado em \url{https://github.com/mbreese/swalign}.}

% subsection implementa_o_do_algoritmo (end)
