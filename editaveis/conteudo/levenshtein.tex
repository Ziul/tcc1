\section{Levenshtein} % (fold)
\label{sec:leveinstein}

Uma das abordagens proposta neste trabalho são baseadas nos trabalhos de \textit{Vladimir Levenshtein}. Atuante na área de teoria da informação, código de correção de erros e design combinacional, Levenshtein publicou 1965 na [REVISTA] aquele que viria a ser o trabalho pelo qual seria reconhecido dês de então: \textit{Binary codes with correction of drooping and insertion of the symbol 1}\cite{levenshtein1965}. Republicado um resumo em 1966 com o titulo de \textit{Binary codes capable of correcting deletions, insertions and reversals}\cite{levenshtein1966}, a obra tratava do calculo da distância entre duas \textit{strings}, ou seja, dada as \textit{strings} \textbf{A} e \textbf{B} a quantidade mínima de operações necessárias para, partindo da \textit{string} \textbf{A}, seja possível chegar na \textit{string} \textbf{B}. Entende-se por operações a inserção, remoção e substituição de um carácter.

\subsection{Exemplo de cálculo de distancia entre \textit{strings}} % (fold)
\label{sub:exemplo_de_c_lculo_de_distancia_entre_it}

Para uma maior compreensão do que se trata uma operação e distancia entre strings, tomemos como exemplos duas palavras, \textbf{cantar} e \textbf{dançar} e vamos proceder com algumas operações para, partindo da primeira, alcançar a segunda palavra.


\begin{enumerate}[start=0]
	\item cantar
	\item dantar \textit{Substituição de 'c' por 'd'}
	\item dançar \textit{Substituição de 't' por 'ç'}
\end{enumerate}

Como podemos observar, foram necessárias duas operações para alcançar \textbf{dançar}, sendo assim, a distância entre elas é $2$.
É importante observar observar que, mesmo que duas letras iguais e consecutivas sejam substituídas por outras duas mesmas letras, serão creditadas duas operações, por exemplo as palavras \textbf{correr} e \textbf{Potter}:

\begin{enumerate}[start=0]
	\item correr
	\item Porrer \textit{Substituição de 'c' por 'P'}
	\item Potrer \textit{Substituição do primeiro 'r' por 't'}
	\item Potter \textit{Substituição do segundo 'r' por 't'}
\end{enumerate}

Precisamos então de 3 operações para alcançar \textbf{Potter} partindo de \textbf{correr}. Toma-se do pressuposto que o algoritmo é \textit{case-sensitive} e sua ordenação é baseada na tabela \textit{ASCII}.

\subsection{Algoritmo da distancia} % (fold)
\label{sub:algoritmo_da_distancia}

Nos trabalhos de Levenshtein em em 1965 e 1966, são apesentados apenas as possibilidades de localização do caractere perdido, ou do bit perdido na sequencia binaria, porém, tamanha descrição apresentada para a identificação do caractere apresentada no trabalho de 1966\cite{levenshtein1966} possibilitaram a implementação do algoritmo no decorrer da historia, tendo uma das sua representações a seguinte forma:


\begin{lstlisting}[language=Python,label=damerau_levenshtein_distance_py,caption={Implementação da distância de Levenshtein}]
def calculate_distance(s1,s2):
    d = {}
    lenstr1 = len(s1)
    lenstr2 = len(s2)
    for i in xrange(-1,lenstr1+1):
        d[(i,-1)] = i+1
    for j in xrange(-1,lenstr2+1):
        d[(-1,j)] = j+1
 
    for i in xrange(lenstr1):
        for j in xrange(lenstr2):
            if s1[i] == s2[j]:
                cost = 0
            else:
                cost = 1
            d[(i,j)] = min(
                           d[(i-1,j)] + 1, # deletion
                           d[(i,j-1)] + 1, # insertion
                           d[(i-1,j-1)] + cost, # substitution
                          )

    return d[lenstr1-1,lenstr2-1]
\end{lstlisting}

Como pode-se observar,  o calculo da distancia é realizado em cima de operações com uma matriz de distancias. Isso possibilitou a evolução das validações de distancias com a paralelização das operações com uso de de GPU ou FPGA, obtendo-se ganhos de operações nas casas de centenas.