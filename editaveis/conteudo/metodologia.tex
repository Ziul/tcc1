\chapter{Metodologia} % (fold)
\label{cha:metodologia}

% - Levantamento de hipoteses x
% - Levantamento de algoritmos de comparação de strings
% - Desenvolvimento de prototipos para os algoritmos selecionados
% - Comparação de resultado dos métodos
% 	# Gerar tabelas x
% 	# Gerar gráficos
% 	- apt-cache
% 	- exact match
% 	- leveinsthein
% 	- align

\section{Visualização do problema} % (fold)
\label{sub:visualiza_o_do_problema}

Dentre os diversos gerenciadores de pacotes utilizados atualmente para a instalação de pacotes em distribuições Linux, podemos apontar alguns que se destacam por virem já instalados de padrão nas distribuições mais comuns hoje.

Porém, todos estes gerenciadores possuem um problema em comum. A apresentação de resultados de busca não é personalizável, não permitindo classificar a listagem de pacotes por nome por exemplo, nem fazem \textit{matching} de aproximação  entre o pacote pesquisado e os possíveis candidatos. Desta forma a procura por um pacote do qual não se tenha o nome correto pode se tornar incrivelmente cansativa e desgastante.

\section{Impondo limites} % (fold)
\label{sub:impondo_limites}


Este trabalho visa apresentar alternativas de busca de pacotes que possam vir a facilitar  o processo de busca. Para tal, será usado como ponto de referência o gerenciador \textit{APT}, devido a sua maior popularidade.

O \textit{APT} possui como repositório oficial o \url{git://anonscm.debian.org/apt/apt.git}, apesar de também ser hospedado em outros repositórios, tal como o \url{https://github.com/Debian/apt}. Neste trabalho estaremos trabalhando em busca de soluções que venha a contribuir e enriquecer a aplicação \href{https://github.com/Debian/apt/blob/debian/experimental/cmdline/apt-cache.cc}{apt-cache}, em especial o parâmetro \textbf{search}. Segundo podemos observar o código disponibilizado do \textit{APT}, sua apresentação de resultados é ordenada de acordo com as seguintes regras:

%% FIXME: Ordem com que o apt-cache ordena as saídas
\begin{enumerate}
	\item  \textbf{FIXME: Ordem com que o apt-cache ordena as saídas}
\end{enumerate}

Mais informações sobre como é o processo \textit{default} de ordenação do \textit{APT} será descrito na sessão \ref{sec:apt}.


\section{Métodos alternativos de ordenação} % (fold)
\label{sec:m_todos_alternativos_de_ordena_o}

No intuito de apresentar alternativas de ordenação de pacotes à atual situação, foram verificadas algoritmos de comparação de strings baseados em artigos de ordenação, classificação e seleção.

Dentre os algoritmos de comparação de \textit{strings}, serão abordados os seguintes:

\begin{itemize}
	\item \textbf{Levenshtein}\\
	Trata do calculo da distância entre duas \textit{strings}. Será visto mais detalhadamente na sessão \ref{sec:leveinstein} deste documento.
%
	\item \textbf{Damerau-Levenshtein}\\
	Variação do método de Levenshtein, a distância de Damerau-Levenshtein consegue abordar alguns pontos de aproximação, como desordem de caracteres. Será visto mais detalhadamente na sessão \ref{sec:damerau_levenshtein} deste documento.
%
	\item \textbf{Smith-Waterman}\\
	Algoritmo de alinhamento de caracteres para determinação de regiões semelhantes entre duas \textit{strings}. Será visto mais detalhadamente na sessão \ref{sec:smith_waterman} deste documento.

\end{itemize}

\subsection{Implementação dos algoritmos} % (fold)
\label{sub:implementa_o_dos_algoritmos}

A primeiro momento, este trabalho tem como objetivo a apresentação de modelos distintos de ordenação no intuito de evidenciar a carência dos gerenciadores de pacotes atuais em apresentação de resultados de pesquisa por pacotes. Consciente do objetivo deste trabalho, será evitado o desenvolvimento dos códigos de algoritmo de ordenação, visto que o objetivo não é a implementação dos algoritmos, mas a comparação dos resultados obtidos através dos algoritmos de ordenação e suas performances nos estados atuais de implementação. Assim, serão utilizadas bibliotecas já implementadas dos algoritmos de ordenação, citando suas respectivas referências e créditos.

Para uma simplificação dos objetivos, foi adotada a linguagem \textit{Python} por apresentar uma interface de interação com o gerenciador de pacote escolhido para uso neste trabalho, \nameref{sec:apt}. Partindo deste ponto, foram localizados algoritmos de ordenação de \textit{strings} disponibilizados no \href{https://pypi.python.org/}{PyPI}.

Foram selecionados então os seguintes algoritmos:

\begin{itemize}
	\item \href{https://pypi.python.org/pypi/swalign/}{swalign}\\
	Pacote que oferece o algoritmo de ordenação de \textit{Smith-Waterman}. Utilizada a versão $0.3.1$.
	\item \href{https://pypi.python.org/pypi/python-Levenshtein/}{python-Levenshtein}\\
	Pacote que oferece o algoritmo de ordenação de \textit{Levenshtein}.Utilizada a versão $0.11.2$.
	\item \href{https://pypi.python.org/pypi/pyxDamerauLevenshtein/}{pyxDamerauLevenshtein}\\
	Pacote que oferece o algoritmo de ordenação de \textit{Damerau-Levenshtein}. Utilizada a versão $1.2$.
\end{itemize}

Também foi utilizado o algoritmo de \textit{match} exato, o qual se idealiza na necessidade de ter a \textit{string} procurada presente no nome do pacote. Esta busca já é implementada no processo de \textit{search} do APT, porém ela aparece em segundo patamar de ordenação. Assim, foi implementado também um método de ordenação baseada apenas no \textit{match} exato  da \textit{string} de entrada pelo nome do pacote.

% subsection implementa_o_dos_algoritmos (end)

% section m_todos_alternativos_de_ordena_o (end)