\section{Metodologia} % (fold)
\label{sec:metodologia}

% - Levantamento de hipoteses
% - Levantamento de algoritmos de comparação de strings
% - Desenvolvimento de prototipos para os algoritmos selecionados
% - Comparação de resultado dos métodos
% 	# Gerar tabelas
% 	# Gerar gráficos
% 	- apt-cache
% 	- exact match
% 	- leveinsthein
% 	- align

\subsection{Visualização do problema} % (fold)
\label{sub:visualiza_o_do_problema}

Dentre os diversos gerenciadores de pacotes utilizados atualmente para a instalação de pacotes em distribuições Linux, podemos apontar alguns que se destacam por virem já instalados de padrão nas distribuições mais comuns hoje. São eles:

\begin{enumerate}
	\item Apt
	\begin{itemize}
		\item Gerenciador amplamente utilizado em distribuições oriundas da Debian (\textit{Debian-like}). Provavelmente o gerenciador mais comum atualmente devido a popularidade de distribuições como \textbf{Ubuntu}, \textbf{Mint} e \textbf{Debian}.
	\end{itemize}
	\item Yum
	\begin{itemize}
		\item Gerenciador padrão  das distribuições oriundas da \textbf{RedHat}, como o \textbf{Fedora} e \textbf{CentOS}.
	\end{itemize}
	\item Portage
	\begin{itemize}
		\item Gerenciador padrão da distribuição Gentoo e suas ramificações.
	\end{itemize}
	\item Pacman
	\begin{itemize}
		\item Dentre os gerenciadores de pacotes, é provalmente o mais recente dentre os mais comuns. Como não poderia deixar de ser, é  o gerenciador de pacotes de uma ramificação recente dentre distribuições Linux que vem conseguindo ganhar seu espaço entre distribuições mais tradicionais, a \textbf{Arch}, lançada em março de 2002.
	\end{itemize}
\end{enumerate}

Porém, todos estes gerenciadores possuem um problema em comum. A apresentação de resultados de busca não é personalizavel, permitindo classificar a listagem de pacotes por nome por exemplo, nem fazem \textit{matching} de aproximação  entre o pacote pesquisado e os possiveis candidatos.

