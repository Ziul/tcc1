\chapter{Metodologia} % (fold)
\label{sec:metodologia}

% - Levantamento de hipoteses x
% - Levantamento de algoritmos de comparação de strings
% - Desenvolvimento de prototipos para os algoritmos selecionados
% - Comparação de resultado dos métodos
% 	# Gerar tabelas x
% 	# Gerar gráficos
% 	- apt-cache
% 	- exact match
% 	- leveinsthein
% 	- align

\section{Visualização do problema} % (fold)
\label{sub:visualiza_o_do_problema}

Dentre os diversos gerenciadores de pacotes utilizados atualmente para a instalação de pacotes em distribuições Linux, podemos apontar alguns que se destacam por virem já instalados de padrão nas distribuições mais comuns hoje. São eles:

\begin{enumerate}
	\item Apt
	\begin{itemize}
		\item Gerenciador amplamente utilizado em distribuições oriundas da Debian (\textit{Debian-like}). Provavelmente o gerenciador mais comum atualmente devido a popularidade de distribuições como \textbf{Ubuntu}, \textbf{Mint} e \textbf{Debian}.
	\end{itemize}
	\item Yum
	\begin{itemize}
		\item Gerenciador padrão  das distribuições oriundas da \textbf{RedHat}, como o \textbf{Fedora} e \textbf{CentOS}.
	\end{itemize}
	\item Portage
	\begin{itemize}
		\item Gerenciador padrão da distribuição \textbf{Gentoo} e suas ramificações. Escrito em \textit{Python} e \textit{Bash}, normalmente o gerenciador é utilizado através da ferramenta \textit{emerge}.
	\end{itemize}
	\item Pacman
	\begin{itemize}
		\item Dentre os gerenciadores de pacotes, é provavelmente o mais recente dentre os mais comuns. Como não poderia deixar de ser, é  o gerenciador de pacotes de uma ramificação recente dentre distribuições Linux que vem conseguindo ganhar seu espaço entre distribuições mais tradicionais, a \textbf{Arch}, lançada em março de 2002.
	\end{itemize}
\end{enumerate}

Porém, todos estes gerenciadores possuem um problema em comum. A apresentação de resultados de busca não é personalizável, não permitindo classificar a listagem de pacotes por nome por exemplo, nem fazem \textit{matching} de aproximação  entre o pacote pesquisado e os possíveis candidatos. Desta forma a procura por um pacote do qual não se tenha o nome correto pode se tornar incrivelmente cansativa e desgastante.

\section{Impondo limites} % (fold)
\label{sub:impondo_limites}


Este trabalho visa apresentar alternativas de busca de pacotes que possam vir a facilitar  o processo de busca. Para tal, será usado como ponto de referência o gerenciador \textit{APT}, devido a sua maior popularidade.

O \textit{APT} possui como repositório oficial o \url{git://anonscm.debian.org/apt/apt.git}, apesar de também ser hospedado em outros repositórios, tal como o \url{https://github.com/Debian/apt}. Neste trabalho estaremos trabalhando em busca de soluções que venha a contribuir e enriquecer a aplicação \href{https://github.com/Debian/apt/blob/debian/experimental/cmdline/apt-cache.cc}{apt-cache}, em especial o parâmetro \textbf{search}. Segundo podemos observar o código disponibilizado do \textit{APT}, sua apresentação de resultados é ordenada de acordo com as seguintes regras:

%% FIXME: Ordem com que o apt-cache ordena as saídas
\begin{enumerate}
	\item 
\end{enumerate}

Mais informações sobre como é o processo \textit{default} de ordenação do \textit{APT} será descrito no capitulo \ref{cha:apt}.
