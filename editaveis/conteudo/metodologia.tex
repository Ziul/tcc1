\chapter{\nmu Metodologia} % (fold)
\label{cha:metodologia}

% Etapas em que o trabalho foi realizado
% 	levantamento bibliografico
% 	codificação
% 	testes
% 	levantamento de algoritmos
% ferramentas utilizadas
% ustificjustificativa das escolha dos metodos/algoritmos
% escolha dos pacotes para teste

Observado a necessidade de métodos que apresentassem resultados mais intuitivos ao usuário, duas escolhas deviam ser tomadas para objetividade do estudo: a escolha da plataforma e gerenciador a adotado para base do estudo e qual linguagem seria adotada para o trabalho.

Dentre os grandes grupos que a comunidade Linux se divide, distribuições oriundas do Debian e do Slackware são as predominantes. Dentre estas duas e suas ramificações, o Ubuntu é uma das distribuições mais indicadas e adotadas por pessoas que estão adentrando no mundo Linux. Desta forma o \textit{APT} torna-se o candidato escolhido dentro o grupo de gerenciadores de repositórios de pacotes.


Para uma agilização na obtenção de resultados para os objetivos, foi adotada a linguagem \textit{Python} por apresentar uma interface de interação com o gerenciador de pacote escolhido para uso neste trabalho, \nameref{subs:apt} e por ser de fácil prototipagem de código ao proporcionar um rápido e enxuto desenvolvimento de código. 

Partindo deste ponto, foram localizados algoritmos de ordenação de \textit{strings} apresentados em artigos e revistas cientificas. Por ser uma área estudada ate mesmo antes da modernização e popularização de computadores sejam por fins militares ou puramente científicos, este foi um dos pontos de difícil escolha de métodos devido a vasta diversidade da qual se é possível encontrar. Guiados por estudos recentes como o de Xu \cite{xu2013bit} com processamento usando GPUs ou mais antigos como o de Galil \cite{galil1988data}, um dos nomes que normalmente eram citados era o de Levenshtein e seu trabalho de 1966 \cite{levenshtein1966}. Este algoritmo por si só apontou outras duas alternativas, sendo uma delas uma variação do primeiro para maior exatidão de resultados. 

Devido o interesse no estudo seja apresentar alternativas para o modelo de busca apresentado hoje no gerenciador e não novos algoritmos de aproximação de \textit{string}, foram priorizados algoritmos de implementação pouco rebuscada, que apresentassem bons resultados de ordenação de \textit{strings} e que já estivessem implementados em linguagem \textit{Python}. Desta forma, foram escolhidos três algoritmos disponibilizados no \href{https://pypi.python.org/}{PyPI}. São eles:

\begin{itemize}
	\item \href{https://pypi.python.org/pypi/swalign/}{swalign}\\
	Pacote que oferece o algoritmo de ordenação de \textit{Smith-Waterman}. Utilizada a versão $0.3.1$.
	\item \href{https://pypi.python.org/pypi/python-Levenshtein/}{python-Levenshtein}\\
	Pacote que oferece o algoritmo de ordenação de \textit{Levenshtein}.Utilizada a versão $0.11.2$.
	\item \href{https://pypi.python.org/pypi/pyxDamerauLevenshtein/}{pyxDamerauLevenshtein}\\
	Pacote que oferece o algoritmo de ordenação de \textit{Damerau-Levenshtein}. Utilizada a versão $1.2$.
\end{itemize}

Também foi utilizado o algoritmo de \textit{match} exato, o qual se idealiza na necessidade de ter a \textit{string} procurada presente no nome do pacote. Esta busca já é implementada no processo de \textit{search} do APT, porém ela aparece em segundo patamar de ordenação. Assim, foi implementado também um método de ordenação baseada apenas no \textit{match} exato  da \textit{string} de entrada pelo nome do pacote.

% subsection implementa_o_dos_algoritmos (end)

% section m_todos_alternativos_de_ordena_o (end)