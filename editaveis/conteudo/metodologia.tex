\chapter{\nmu Metodologia} % (fold)
\label{cha:metodologia}

% Etapas em que o trabalho foi realizado
% 	levantamento bibliografico
% 	codificação
% 	testes
% 	levantamento de algoritmos
% ferramentas utilizadas
% ustificjustificativa das escolha dos metodos/algoritmos
% escolha dos pacotes para teste


\section{Métodos alternativos de ordenação} % (fold)
\label{sec:m_todos_alternativos_de_ordena_o}

No intuito de apresentar alternativas de ordenação de pacotes à atual situação, foram verificadas algoritmos de comparação de strings baseados em artigos de ordenação, classificação e seleção.

Dentre os algoritmos de comparação de \textit{strings}, serão abordados os seguintes:

\begin{itemize}
	\item \textbf{Levenshtein}\\
	Trata do calculo da distância entre duas \textit{strings}. Será visto mais detalhadamente na sessão \ref{sec:leveinstein} deste documento.
%
	\item \textbf{Damerau-Levenshtein}\\
	Variação do método de Levenshtein, a distância de Damerau-Levenshtein consegue abordar alguns pontos de aproximação, como desordem de caracteres. Será visto mais detalhadamente na sessão \ref{sec:damerau_levenshtein} deste documento.
%
	\item \textbf{Smith-Waterman}\\
	Algoritmo de alinhamento de caracteres para determinação de regiões semelhantes entre duas \textit{strings}. Será visto mais detalhadamente na sessão \ref{sec:smith_waterman} deste documento.

\end{itemize}

\subsection{Implementação dos algoritmos} % (fold)
\label{sub:implementa_o_dos_algoritmos}

A primeiro momento, este trabalho tem como objetivo a apresentação de modelos distintos de ordenação no intuito de evidenciar a carência dos gerenciadores de pacotes atuais em apresentação de resultados de pesquisa por pacotes. Consciente do objetivo deste trabalho, será evitado o desenvolvimento dos códigos de algoritmo de ordenação, visto que o objetivo não é a implementação dos algoritmos, mas a comparação dos resultados obtidos através dos algoritmos de ordenação e suas performances nos estados atuais de implementação. Assim, serão utilizadas bibliotecas já implementadas dos algoritmos de ordenação, citando suas respectivas referências e créditos.

Para uma simplificação dos objetivos, foi adotada a linguagem \textit{Python} por apresentar uma interface de interação com o gerenciador de pacote escolhido para uso neste trabalho, \nameref{sec:apt}. Partindo deste ponto, foram localizados algoritmos de ordenação de \textit{strings} disponibilizados no \href{https://pypi.python.org/}{PyPI}.

Foram selecionados então os seguintes algoritmos:

\begin{itemize}
	\item \href{https://pypi.python.org/pypi/swalign/}{swalign}\\
	Pacote que oferece o algoritmo de ordenação de \textit{Smith-Waterman}. Utilizada a versão $0.3.1$.
	\item \href{https://pypi.python.org/pypi/python-Levenshtein/}{python-Levenshtein}\\
	Pacote que oferece o algoritmo de ordenação de \textit{Levenshtein}.Utilizada a versão $0.11.2$.
	\item \href{https://pypi.python.org/pypi/pyxDamerauLevenshtein/}{pyxDamerauLevenshtein}\\
	Pacote que oferece o algoritmo de ordenação de \textit{Damerau-Levenshtein}. Utilizada a versão $1.2$.
\end{itemize}

Também foi utilizado o algoritmo de \textit{match} exato, o qual se idealiza na necessidade de ter a \textit{string} procurada presente no nome do pacote. Esta busca já é implementada no processo de \textit{search} do APT, porém ela aparece em segundo patamar de ordenação. Assim, foi implementado também um método de ordenação baseada apenas no \textit{match} exato  da \textit{string} de entrada pelo nome do pacote.

% subsection implementa_o_dos_algoritmos (end)

% section m_todos_alternativos_de_ordena_o (end)