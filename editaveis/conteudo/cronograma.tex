\chapter{Cronograma}
\label{cha:cronograma}

\begin{ganttchart}[
	%canvas/.style={fill=yellow!25,draw=blue, dashed, very thick}
	hgrid=false,
	vgrid={*1{blue, dotted}},
	%vgrid={*2{red}, *1{green}, *{10}{blue, dashed}}
	x unit = 8mm,
]{35}{47}%{2014-08-28}{2013-08-13}}
\gantttitle{Cronograma}{13} \\
%
\ganttgroup{Fundamentação}{35}{40} \\
\ganttbar{1}{35}{36} \\ % Levantar documentação oficial das distribuições
\ganttbar{2}{35}{36} \\ % Levantar gerenciadores das principais distribuições Linux
\ganttbar{3}{35}{36} \\ % Configurar maquinas Vagrant
\ganttbar{4}{35}{38} \\ % Estudar documentação de String Matching
%
\ganttgroup{Prototipação}{40}{43} \\
\ganttbar{5}{40}{42} \\ % Escrever primeiro protótipo
\ganttbar{6}{40}{42} \\ % Testar com 5 pacotes
\ganttbar{7}{41}{43} \\ % Escrever segundo protótipo
\ganttbar{8}{42}{43} \\ % Escrever terceiro protótipo
%
\ganttgroup{Documentação}{38}{47} \\
\ganttbar{9}{38}{40} \\ % Estruturar documento do TCC
\ganttbar{10}{38}{40} \\ % Escrever fundamentação teórica
\ganttbar{11}{43}{45} \\ % Escrita do TCC
\ganttbar{12}{45}{46} \\ % Revisão do TCC
\ganttbar{13}{45}{46} \\ % Produção da Apresentação
\ganttbar{14}{47}{47} \\ % Entrega
\end{ganttchart}

\textbf{Legenda:}
\begin{description}
\item [1]  Levantar documentação oficial das distribuições
\item [2]  Levantar gerenciadores das principais distribuições Linux
\item [3]  Configurar maquinas Vagrant
\item [4]  Estudar documentação de String Matching
\item [5]  Escrever primeiro protótipo
\item [6]  Testar com 5 pacotes
\item [7]  Escrever segundo protótipo
\item [8]  Escrever terceiro protótipo
\item [9]  Estruturar documento do TCC
\item [10] Escrever fundamentação teórica
\item [11] Escrita do TCC
\item [12] Revisão do TCC
\item [13] Produção da Apresentação
\item [14] Entrega
\end{description}