% TODO Levar o conteúdo desta seção para a subseção do apt
\subsubsection{APT} % (fold)
\label{subs:apt}

O {\code APT}, \textit{Advanced Package Tool}, é um gerenciador de repositórios  amplamente utilizado em distribuições oriundas da Debian (\textit{Debian-like}).
É provavelmente o gerenciador mais comum atualmente devido a popularidade de distribuições como \textbf{Ubuntu}, \textbf{Mint} e \textbf{Debian}.

Por ser um \textit{gerenciador de repositórios}, o {\code APT} trabalha unicamente com os arquivos \textit{.deb} que possuem seus respectivos endereços registrados no arquivo {\code/etc/apt/sources. list}, possibilitando o uso de diretórios remotos ou locais para a disponibilização dos pacotes a serem gerenciados. Requisitado a instalação de um pacote ao {\code APT}, será feito o \textit{download} do pacote e  suas respectivas dependências e repassado ao {\code dpkg} para que seja realizado o processo de instalação.

%(data de criação, autores, distros, principais vantagens/características, gerenciador de pacotes que roda por baixo (citações das documentações)

% Como introduzido em \ref{sub:visualiza_o_do_problema}, o \textit{APT} possui como repositório oficial o \url{git://anonscm.debian.org/apt/apt.git}, apesar de também ser hospedado em outros repositórios, tal como o \url{https://github.com/Debian/apt}. Uma observação mais atenta no \href{https://github.com/Debian/apt/blob/debian/experimental/cmdline/apt-cache.cc}{apt-cache.cc}, disponibilizado no repositório nos esclarece como o \textit{apt-cache} procede na ordenação dos pacotes para impressão.

%\begin{lstlisting}[language=C++]
%LocalitySort(&DFList->Df,Cache->HeaderP->GroupCount,sizeof(*DFList));
%\end{lstlisting}
