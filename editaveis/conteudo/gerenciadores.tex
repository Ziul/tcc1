\section{Gerenciadores de pacotes} % (fold)
\label{sec:distribui_es_abordadas}

\subsection{Contextualização} % (fold)
\label{sec:breve_descri_o}

Desde o lançamento das distribuições Debian e Slackware em 1993, diversas distribuições tiveram seu ciclo de vida realizados por completo, surgindo, apresentando e amadurecendo sua filosofia por trás da distribuição, gerando novas ramificações e por fim sendo abandonadas. Hoje, se escolhermos alguma distribuição Linux dentre as inúmeras existentes, provavelmente ela será descendente das principais distribuições hoje no mercado: \textit{Debian, Red Hat, Slackware} ou \textit{Arch}
% Adicionar a Gentoo dentre as principais distribuições?

Em sistemas operacionais Linux, a muito foi definido uma padronização para distribuições de aplicações para o sistemas, chamados pacotes. A filosofia por trás da comunidade livre e o interesse em buscar novas formas de armazenamento e distribuição de pacotes fez com que surgissem diversos \textit{gerenciadores de pacotes}, especializados para cada distribuição.

Devido grande parte das distribuições hoje existentes serem ramificações das distribuições \textit{Debian, Red Hat, Slackware} ou \textit{Arch},os diversos gerenciadores de pacotes hoje disponíveis também possuem muitas similaridades entre eles.
Estas distribuições representam o berço de grande parte das distribuições existentes hoje. Assim, este trabalho buscará apresentar resultados que possam ser facilmente convertido para os demais gerenciadores, sempre visando a maior compatibilidade possível com o maior número de distribuições disponíveis.

\subsection{Gerenciadores de pacotes e Gerenciadores de Repositórios} % (fold)
\label{sec:gerenciadores}

% TODO Definir e diferenciar gerenciadores de pacotes e gerenciadores de
% repositórios

Tendo em vista que este trabalho busca apresentar melhores soluções para a ordenação de pacotes na resposta recebida em uma pesquisa realizada em um gerenciador de pacotes, devemos inicialmente estudar os possíveis candidatos a ser estudado no trabalho.
Desta forma, os gerenciadores de pacotes que serão levados em consideração neste trabalho serão:

% TODO Transformar cada item da descrição em uma subsubsection, com dois ou
% três parágrafos (data de criação, autores, distros, principais vantagens/
% características, gerenciador de pacotes que roda por baixo (citações das documentações)
\begin{description}
	\item [Apt] Gerenciador amplamente utilizado em distribuições oriundas da Debian (\textit{Debian-like}). Provavelmente o gerenciador mais comum atualmente devido a popularidade de distribuições como \textbf{Ubuntu}, \textbf{Mint} e \textbf{Debian}.
	\item [Yum] Gerenciador padrão  das distribuições oriundas da \textbf{RedHat}, como o \textbf{Fedora} e \textbf{CentOS}.
	\item [Portage] Gerenciador padrão da distribuição \textbf{Gentoo} e suas ramificações. Escrito em \textit{Python} e \textit{Bash}, normalmente o gerenciador é utilizado através da ferramenta \textit{emerge}.
	\item [Pacman] Dentre os gerenciadores de pacotes, é provavelmente o mais recente dentre os mais comuns. Como não poderia deixar de ser, é  o gerenciador de pacotes de uma ramificação recente dentre distribuições Linux que vem conseguindo ganhar seu espaço entre distribuições mais tradicionais, a \textbf{Arch}, lançada em março de 2002.
\end{description}


\subsubsection{Comandos básicos} % (fold)
\label{sub:apt}

% TODO Escrever uma intro (3 a 4 parágrafos) descrevendo rapidamente os
% principais comandos (instalação, remoção, busca e consulta). Depois
% montar uma tabela de comandos por gerenciador
Dentre os comandos que a ferramenta disponibiliza, os comandos básicos (procura por pacote, instalar pacote e remover pacote) possuem a seguinte sintaxe:

\begin{itemize}
	\item Instalar
	\begin{lstlisting}[numbers=none,commentstyle=\color{black}]
	# apt-get install  <package>
	\end{lstlisting}
	\item Remover
	\begin{lstlisting}[numbers=none,commentstyle=\color{black}]
	# apt-get purge  <package>
	\end{lstlisting}
	\item Procurar pacote
	\begin{lstlisting}[numbers=none,commentstyle=\color{black}]
	# apt-cache search  <package>
	\end{lstlisting}
\end{itemize}


\subsubsubsection{Yum} % (fold)
\label{sub:yum}

Dentre os comandos que a ferramenta disponibiliza, os comandos básicos (procura por pacote, instalar pacote e remover pacote) possuem a seguinte sintaxe:

\begin{itemize}
	\item Instalar
	\begin{lstlisting}[numbers=none,commentstyle=\color{black}]
	# yum install  <package>
	\end{lstlisting}
	\item Remover
	\begin{lstlisting}[numbers=none,commentstyle=\color{black}]
	# yum purge  <package>
	\end{lstlisting}
	\item Procurar pacote
	\begin{lstlisting}[numbers=none,commentstyle=\color{black}]
	# yum search  <package>
	\end{lstlisting}
\end{itemize}


\subsubsubsection{Pacman} % (fold)
\label{sub:pacan}

Dentre os comandos que a ferramenta disponibiliza, os comandos básicos (procura por pacote, instalar pacote e remover pacote) possuem a seguinte sintaxe:

\begin{itemize}
	\item Instalar
	\begin{lstlisting}[numbers=none,commentstyle=\color{black}]
	# pacman -S  <package>
	\end{lstlisting}
	\item Remover
	\begin{lstlisting}[numbers=none,commentstyle=\color{black}]
	# pacman -Rc  <package>
	\end{lstlisting}
	\item Procurar pacote$^*$
	\begin{lstlisting}[numbers=none,commentstyle=\color{black}]
	# pacman -Si  <package>
	# pacman -Qi  <package>
	\end{lstlisting}
\end{itemize}


\subsubsubsection{Zypper} % (Zypper)
\label{sub:zypper}

Dentre os comandos que a ferramenta disponibiliza, os comandos básicos (procura por pacote, instalar pacote e remover pacote) possuem a seguinte sintaxe:

\begin{itemize}
	\item Instalar
	\begin{lstlisting}[numbers=none,commentstyle=\color{black}]
	# zypper install  <package>
	\end{lstlisting}
	\item Remover
	\begin{lstlisting}[numbers=none,commentstyle=\color{black}]
	# zypper purge  <package>
	\end{lstlisting}
	\item Procurar pacote
	\begin{lstlisting}[numbers=none,commentstyle=\color{black}]
	# zypper search  <package>
	\end{lstlisting}
\end{itemize}
