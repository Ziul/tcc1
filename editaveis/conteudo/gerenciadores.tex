\section{Gerenciadores de pacotes} % (fold)
\label{sec:distribui_es_abordadas}

\subsection{Contextualização} % (fold)
\label{sec:breve_descri_o}

Desde o lançamento das distribuições Debian e Slackware em 1993, diversas distribuições tiveram seu ciclo de vida realizado por completo, apresentando e amadurecendo a filosofia por trás da distribuição, gerando novas ramificações e por fim sendo abandonadas. Hoje, se escolhermos alguma distribuição Linux dentre as inúmeras existentes, provavelmente ela será descendente de uma das principais distribuições: \textit{Debian, Red Hat, Slackware} ou \textit{Arch}.
% Adicionar a Gentoo dentre as principais distribuições?

Em sistemas operacionais Linux, há muito foi proposto uma alternativa para distribuições de aplicações para o sistemas, chamados pacotes. A filosofia por trás da comunidade livre e o interesse em buscar novas formas de armazenamento e distribuição de pacotes fez com que surgissem diversos \textit{gerenciadores de pacotes}, especializados para cada distribuição.

Devido grande parte das distribuições hoje existentes serem ramificações das distribuições \textit{Debian, Red Hat, Slackware} ou \textit{Arch}, os diversos gerenciadores de pacotes hoje disponíveis também possuem muitas similaridades entre eles.
Assim, este trabalho buscará apresentar resultados que possam ser aplicadas em qualquer gerenciador, sempre visando a maior compatibilidade possível com o maior número de distribuições disponíveis.


% TODO: Acrescentar um parágrafo citando alguns gerenciadores de pacotes e formatos (dpkg/deg, rpm, etc).

\subsection{Gerenciadores de pacotes e Gerenciadores de Repositórios} % (fold)
\label{sec:gerenciadores}

% TODO Definir e diferenciar gerenciadores de pacotes e gerenciadores de
% repositórios

Tendo em vista que este trabalho busca apresentar uma melhor solução para a ordenação de pacotes listados na resposta recebida em uma pesquisa realizada em um gerenciador de repositórios, devemos inicialmente elencar os possíveis candidatos a serem utilizados no trabalho.
Desta forma, os gerenciadores de repositórios que serão levados em consideração neste trabalho serão:

% TODO Transformar cada item da descrição em uma subsubsection, com dois ou
% três parágrafos (data de criação, autores, distros, principais vantagens/
% características, gerenciador de pacotes que roda por baixo (citações das documentações)
% \begin{description}
% 	\item [Apt] Gerenciador amplamente utilizado em distribuições oriundas da Debian (\textit{Debian-like}). Provavelmente o gerenciador mais comum atualmente devido a popularidade de distribuições como \textbf{Ubuntu}, \textbf{Mint} e \textbf{Debian}.
% 	\item [Yum] Gerenciador padrão  das distribuições oriundas da \textbf{RedHat}, como o \textbf{Fedora} e \textbf{CentOS}.
% 	\item [Portage] Gerenciador padrão da distribuição \textbf{Gentoo} e suas ramificações. Escrito em \textit{Python} e \textit{Bash}, normalmente o gerenciador é utilizado através da ferramenta \textit{emerge}.
% 	\item [Pacman] Dentre os gerenciadores de pacotes, é provavelmente o mais recente dentre os mais comuns. Como não poderia deixar de ser, é  o gerenciador de pacotes de uma ramificação recente dentre distribuições Linux que vem conseguindo ganhar seu espaço entre distribuições mais tradicionais, a \textbf{Arch}, lançada em março de 2002.
% \end{description}
% TODO Levar o conteúdo desta seção para a subseção do apt
\subsubsection{APT} % (fold)
\label{subs:apt}

O {\code APT}, \textit{Advanced Package Tool}, é um gerenciador de repositórios  amplamente utilizado em distribuições oriundas da Debian (\textit{Debian-like}).
É provavelmente o gerenciador mais comum atualmente devido a popularidade de distribuições como \textbf{Ubuntu}, \textbf{Mint} e \textbf{Debian}.

Por ser um \textit{gerenciador de repositórios}, o {\code APT} trabalha unicamente com os arquivos \textit{.deb} que possuem seus respectivos endereços registrados no arquivo {\code/etc/apt/sources. list}, possibilitando o uso de diretórios remotos ou locais para a disponibilização dos pacotes a serem gerenciados. Requisitado a instalação de um pacote ao {\code APT}, será feito o \textit{download} do pacote e  suas respectivas dependências e repassado ao {\code dpkg} para que seja realizado o processo de instalação.

%(data de criação, autores, distros, principais vantagens/características, gerenciador de pacotes que roda por baixo (citações das documentações)

% Como introduzido em \ref{sub:visualiza_o_do_problema}, o \textit{APT} possui como repositório oficial o \url{git://anonscm.debian.org/apt/apt.git}, apesar de também ser hospedado em outros repositórios, tal como o \url{https://github.com/Debian/apt}. Uma observação mais atenta no \href{https://github.com/Debian/apt/blob/debian/experimental/cmdline/apt-cache.cc}{apt-cache.cc}, disponibilizado no repositório nos esclarece como o \textit{apt-cache} procede na ordenação dos pacotes para impressão.

%\begin{lstlisting}[language=C++]
%LocalitySort(&DFList->Df,Cache->HeaderP->GroupCount,sizeof(*DFList));
%\end{lstlisting}

\subsubsection{YUM} % (fold)
\label{ssub:yum}

O Yum é o gerenciador de repositórios padrão  das distribuições oriundas da \textbf{RedHat}, como o \textbf{Fedora} e \textbf{CentOS}.

\subsubsection{Emerge} % (fold)
\label{ssub:emerge}

O Emerge é comumente usado como o gerenciador de repositórios da distribuição \textbf{Gentoo} e suas ramificações. Normalmente o gerenciador é utilizado através do gerenciador de repositórios \textit{emerge}, este por sua vez escrito em \textit{Python} e \textit{Bash}.
\subsubsection{Pacman} % (fold)
\label{ssub:pacman}

Dentre os gerenciadores de repositórios, o Pacman é provavelmente o mais recente dentre os mais usuais. Como não poderia deixar de ser, é  o gerenciador de repositórios de uma ramificação recente dentre distribuições Linux que vem conseguindo ganhar seu espaço entre distribuições mais tradicionais, a \textbf{Arch}, lançada em março de 2002

\subsubsection{Comandos básicos} % (fold)
\label{subs:comandos_basicos}

% TODO Escrever uma intro (3 a 4 parágrafos) descrevendo rapidamente os
% principais comandos (instalação, remoção, busca e consulta). Depois
% montar uma tabela de comandos por gerenciador

% TODO: Reduzir tamanho da tabela
\begin{table}[htbp]
\caption{Principais comandos dos gerenciadores de repositórios mais populares}
\begin{tabular}{|l|c|c|c|}
\hline
& \textbf{Instalar} & \textbf{Remover} & \textbf{Procurar} \\ \hline
\textbf{APT} & apt-get install <pacote> & apt-get purge <pacote> & apt-cache search <pacote> \\ \hline
\textbf{YUM} & yum install <pacote> & yum purge <pacote> & yum search <pacote> \\ \hline
\textbf{Pacman} & pacman -S <pacote> & pacman -Rc <pacote> & pacman -Si  <pacote> \&\& pacman -Qi  <pacote> \\ \hline
\textbf{Emerge} & emerge <pacote> & emerge remove <pacote> & emerge search <pacote> \\ \hline
\end{tabular}
\label{cmd_gerenciadores}
\end{table}
