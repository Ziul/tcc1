\section{Gerenciadores de pacotes} % (fold)
\label{sec:distribui_es_abordadas}

\subsection{Contextualização} % (fold)
\label{sec:breve_descri_o}

Desde o lançamento das distribuições Debian e Slackware em 1993, diversas distribuições tiveram seu ciclo de vida realizados por completo, surgindo, apresentando e amadurecendo sua filosofia por trás da distribuição, gerando novas ramificações e por fim sendo abandonadas. Hoje, se escolhermos alguma distribuição Linux dentre as inúmeras existentes, provavelmente ela será descendente das principais distribuições hoje no mercado: \textit{Debian, Red Hat, Slackware} ou \textit{Arch}
% Adicionar a Gentoo dentre as principais distribuições?

Em sistemas operacionais Linux, a muito foi definido uma padronização para distribuições de aplicações para o sistemas, chamados pacotes. A filosofia por trás da comunidade livre e o interesse em buscar novas formas de armazenamento e distribuição de pacotes fez com que surgissem diversos \textit{gerenciadores de pacotes}, especializados para cada distribuição.

Devido grande parte das distribuições hoje existentes serem ramificações das distribuições \textit{Debian, Red Hat, Slackware} ou \textit{Arch},os diversos gerenciadores de pacotes hoje disponíveis também possuem muitas similaridades entre eles.
Estas distribuições representam o berço de grande parte das distribuições existentes hoje. Assim, este trabalho buscará apresentar resultados que possam ser facilmente convertido para os demais gerenciadores, sempre visando a maior compatibilidade possível com o maior número de distribuições disponíveis.

\subsection{Gerenciadores de pacotes e Gerenciadores de Repositórios} % (fold)
\label{sec:gerenciadores}

% TODO Definir e diferenciar gerenciadores de pacotes e gerenciadores de
% repositórios

Tendo em vista que este trabalho busca apresentar melhores soluções para a ordenação de pacotes na resposta recebida em uma pesquisa realizada em um gerenciador de pacotes, devemos inicialmente estudar os possíveis candidatos a ser estudado no trabalho.
Desta forma, os gerenciadores de pacotes que serão levados em consideração neste trabalho serão:

% TODO Transformar cada item da descrição em uma subsubsection, com dois ou
% três parágrafos (data de criação, autores, distros, principais vantagens/
% características, gerenciador de pacotes que roda por baixo (citações das documentações)
% \begin{description}
% 	\item [Apt] Gerenciador amplamente utilizado em distribuições oriundas da Debian (\textit{Debian-like}). Provavelmente o gerenciador mais comum atualmente devido a popularidade de distribuições como \textbf{Ubuntu}, \textbf{Mint} e \textbf{Debian}.
% 	\item [Yum] Gerenciador padrão  das distribuições oriundas da \textbf{RedHat}, como o \textbf{Fedora} e \textbf{CentOS}.
% 	\item [Portage] Gerenciador padrão da distribuição \textbf{Gentoo} e suas ramificações. Escrito em \textit{Python} e \textit{Bash}, normalmente o gerenciador é utilizado através da ferramenta \textit{emerge}.
% 	\item [Pacman] Dentre os gerenciadores de pacotes, é provavelmente o mais recente dentre os mais comuns. Como não poderia deixar de ser, é  o gerenciador de pacotes de uma ramificação recente dentre distribuições Linux que vem conseguindo ganhar seu espaço entre distribuições mais tradicionais, a \textbf{Arch}, lançada em março de 2002.
% \end{description}
% TODO Levar o conteúdo desta seção para a subseção do apt
\subsubsection{APT} % (fold)
\label{subs:apt}

O {\code APT}, \textit{Advanced Package Tool}, é um gerenciador de repositórios  amplamente utilizado em distribuições oriundas da Debian (\textit{Debian-like}).
É provavelmente o gerenciador mais comum atualmente devido a popularidade de distribuições como \textbf{Ubuntu}, \textbf{Mint} e \textbf{Debian}.
Projetado inicialmente para uma substituição do gerenciador {\code dselect}, o APT teve suas primeiras \textit{builds} distribuídas via IRC em Agosto de 1998,  sendo integrado ao \textit{Debian} na \textit{release} de Março de 1999 \cite{garbee2008brief}. O  APT pode ser considerado como uma interface para o {\code dpkg}, gerenciando as dependências de um pacote para a instalação e remoção, alem de apresentar uma listagem dos pacotes disponíveis na lista de repositórios. Um dos recursos que o APT realiza de forma transparente para o usuário é a ordenação de pacotes para instalação ou remoção com o uso das chamadas do {\code dpkg}, suprindo as dependências dos pacotes que são requisitos para o pacote ao qual o usuário deseja instalar na maquina. Dentre as desvantagens que o APT pode apresentar, as que se destacam são justamente o fato de estar escrito em C++, que apesar do ganho de performance que apresenta aos seus concorrentes, carrega junto uma maior dificuldade de manutenção devido ao tamanho e complexidade. 

Outro problema do APT é sua fragmentação. Apesar de serem tratados como parte do APT, as aplicações como {\code apt-get} ou {\code apt-cache} são na realidade aplicações que fazem de uso do APT como uma biblioteca para interface, mas são comumente tratadas como aplicações integradas ao APT. Como consequências, o APT possuiu uma das interfaces menos intuitivas quando comparado com gerenciadores como o \textit{YUM} ou o \textit{Portage} justamente por frequentemente serem apresentadas soluções utilizando as aplicações como o {\code apt-get} ao invés do {\code apt}.

Por ser um \textit{gerenciador de repositórios}, o {\code APT} trabalha unicamente com os arquivos \textit{.deb} que possuem seus respectivos endereços registrados no arquivo {\code/etc/apt/sources. list}, possibilitando o uso de diretórios remotos ou locais para a disponibilização dos pacotes a serem gerenciados. Requisitado a instalação de um pacote ao {\code APT}, será feito o \textit{download} do pacote e  suas respectivas dependências e repassado ao {\code dpkg} para que seja realizado o processo de instalação.

O \textit{APT} possui como repositório oficial o \url{git://anonscm.debian.org/apt/apt.git}, apesar de também ser hospedado em outros repositórios, tal como o \url{https://github.com/Debian/apt}. Neste trabalho estaremos trabalhando em busca de soluções que venha a contribuir e enriquecer em especial a aplicação \href{https://github.com/Debian/apt/blob/debian/experimental/cmdline/apt-cache.cc}{apt-cache}, em especial o parâmetro \textbf{search}. Segundo podemos observar o código disponibilizado do \textit{APT}, sua apresentação de resultados é ordenada alfabeticamente de acordo com a lista de repositórios disponíveis.


%(data de criação, autores, distros, principais vantagens/características, gerenciador de pacotes que roda por baixo (citações das documentações)


%\begin{lstlisting}[language=C++]
%LocalitySort(&DFList->Df,Cache->HeaderP->GroupCount,sizeof(*DFList));
%\end{lstlisting}

\subsubsection{YUM} % (fold)
\label{ssub:yum}

O Yum é o gerenciador de repositórios padrão  das distribuições oriundas da \textbf{RedHat}, como o \textbf{Fedora} e \textbf{CentOS}.

\subsubsection{Portage} % (fold)
\label{ssub:emerge}

O Portage é  o gerenciador de repositórios padrão da distribuição \textbf{Gentoo} e suas ramificações \cite{vermeulen2005gentoo}. O Portage é na verdade duas aplicações trabalhando em conjunto para o funcionamento de um todo: o {\code ebuild} que cuida do processo de \textit{building} e instalação dos pacotes enquanto o {\code emerge} providência a interface ao usuário. O Portage foi predominantemente escrito em \textit{Python}.

O Portage, por padrão, não faz de uso de pacotes binários, tais como RPM ou DEB, mas sim a compilação da aplicação durante o processo de instalação. Com resultados o Portage mantem uma alta compatibilidade da aplicação com a maquina e alta flexibilidade com as arquiteturas de suporte, ao custo da demora para a instalação.
Essa flexibilidade permite que o Portage não seja em momento algum dependente do sistema, sendo por vezes utilizado para gerenciar outros sistemas, tais como o BSD, Mac OS e Solaris \cite{ryan2010linux}.
Eventualmente, \textit{forks} do Portage são por vezes apresentados como soluções para sistemas embarcados pelo fato da portabilidade da aplicação depender apenas da presença do compilador \cite{guointegrated}. Todavia, o Portage apresenta também uma solução de instalação com empacotamento de binários através das \textit{flags} {\code --usepkg --getbinpkg} \cite{gentoo-doc}  para instalações mais rápidas.
\subsubsection{Pacman} % (fold)
\label{ssub:pacman}

Dentre os gerenciadores de repositórios, o Pacman é provavelmente o mais recente dentre os mais usuais. Como não poderia deixar de ser, é  o gerenciador de repositórios de uma ramificação recente dentre distribuições Linux que vem conseguindo ganhar seu espaço entre distribuições mais tradicionais, a \textbf{Arch}, lançada em março de 2002

\subsubsection{Comandos básicos} % (fold)
\label{subs:comandos_basicos}

% TODO Escrever uma intro (3 a 4 parágrafos) descrevendo rapidamente os
% principais comandos (instalação, remoção, busca e consulta). Depois
% montar uma tabela de comandos por gerenciador

% % APT
% \begin{itemize}
% 	\item Instalar
% 	\begin{lstlisting}[numbers=none,commentstyle=\color{black}]
% 	# apt-get install  <package>
% 	\end{lstlisting}
% 	\item Remover
% 	\begin{lstlisting}[numbers=none,commentstyle=\color{black}]
% 	# apt-get purge  <package>
% 	\end{lstlisting}
% 	\item Procurar pacote
% 	\begin{lstlisting}[numbers=none,commentstyle=\color{black}]
% 	# apt-cache search  <package>
% 	\end{lstlisting}
% \end{itemize}


% % Yum
% \begin{itemize}
% 	\item Instalar
% 	\begin{lstlisting}[numbers=none,commentstyle=\color{black}]
% 	# yum install  <package>
% 	\end{lstlisting}
% 	\item Remover
% 	\begin{lstlisting}[numbers=none,commentstyle=\color{black}]
% 	# yum purge  <package>
% 	\end{lstlisting}
% 	\item Procurar pacote
% 	\begin{lstlisting}[numbers=none,commentstyle=\color{black}]
% 	# yum search  <package>
% 	\end{lstlisting}
% \end{itemize}


% % Pacman
% \begin{itemize}
% 	\item Instalar
% 	\begin{lstlisting}[numbers=none,commentstyle=\color{black}]
% 	# pacman -S  <package>
% 	\end{lstlisting}
% 	\item Remover
% 	\begin{lstlisting}[numbers=none,commentstyle=\color{black}]
% 	# pacman -Rc  <package>
% 	\end{lstlisting}
% 	\item Procurar pacote$^*$
% 	\begin{lstlisting}[numbers=none,commentstyle=\color{black}]
% 	# pacman -Si  <package>
% 	# pacman -Qi  <package>
% 	\end{lstlisting}
% \end{itemize}


% Emerge