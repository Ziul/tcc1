\section{Gerenciadores de pacotes} % (fold)
\label{sec:distribui_es_abordadas}

\subsection{Contextualização} % (fold)
\label{sec:breve_descri_o}

Desde o lançamento das distribuições \textit{Debian} e \textit{Slackware} em 1993, diversas distribuições tiveram seu ciclo de vida realizado por completo, apresentando e amadurecendo a filosofia por trás da distribuição, gerando novas ramificações e por fim sendo abandonadas. Hoje, se escolhermos alguma distribuição Linux dentre as inúmeras existentes, provavelmente ela será descendente de uma das principais distribuições: \textit{Debian, Red Hat} ou \textit{Slackware}
.% ou \textit{Arch}, 
A \autoref{fig:figuras_linux_timeline_heranca}, no \autoref{long_figures} apresenta um modelo de herança entre as distribuições Linux onde pode-se observar a influência que estas distribuições fazem a um grupo de novas distribuições que se ramificaram delas
\footnote{Apesar da difícil visualização da imagem no documento, o intuito dela aqui é apresentar a vasta quantidade de distribuições que surgiram de bifurcações das distribuições \textit{Debian, Red Hat} e \textit{Slackware}, tal como alterações nelas provavelmente viriam a influenciar as demais distribuições.}.

Em sistemas operacionais Linux, há muito foi proposto uma alternativa para distribuições de aplicações para o sistemas, chamados pacotes. A filosofia por trás da comunidade livre \cite{bretthauer2001open} e o interesse em buscar novas formas de armazenamento e distribuição de pacotes fez com que surgissem diversos \textit{gerenciadores de pacotes}, voltados à determinadas distribuição  cada \cite{beck2002linux}.

Devido grande parte das distribuições hoje existentes serem ramificações das distribuições \textit{Debian, Red Hat} ou \textit{Slackware} 
.% ou \textit{Arch}, 
os diversos gerenciadores de pacotes hoje disponíveis também possuem muitas similaridades entre eles.
É o caso do \textit{dpkg} e o \textit{rpm}, onde as funcionalidades de instalação, remoção e provimento de informações de pacotes esta presente em ambos. Ambos também são escritos predominantemente em \textit{C} e \textit{Perl}. Suas diferenças surgem quando chegamos a averiguar quais pacotes cada aplicação da suporte. O \textit{dpkg} esta voltado para os pacotes \textit{.deb}, enquanto o \textit{rpm} esta voltado para os pacotes \textit{rpm}. Como há uma distinção em como estes pacotes são montados \cite{bailey1997maximum}, há uma incompatibilidade entre as duas aplicações. 

% TODO: Acrescentar um parágrafo citando alguns gerenciadores de pacotes e formatos (dpkg/deg, rpm, etc). %DONE

\subsection{Gerenciadores de pacotes e Gerenciadores de Repositórios} % (fold)
\label{sec:gerenciadores}

% TODO Definir e diferenciar gerenciadores de pacotes e gerenciadores de
% repositórios % DONE

Um ponto de vista importante de ser caracterizado neste trabalho é a diferença entre os termos \textit{ gerenciadores de pacotes} e \textit{gerenciadores de repositórios}.
É importante caracterizar a diferença estes dois termos comumente utilizados como sinônimos devido a sua similaridade. Suas proximidades são tão grandes que na própria documentação oficial do \textit{Debian} não há uma caracterização de \textit{gerenciadores de repositórios}, citando apenas que gerenciadores como o {\code aptitude} ou {\code dselect} dependem do {\code APT}, que por sua vez depende do {\code dpkg} \cite{debian-faq}. Essa cadeia de dependências indicam as camadas de interface, onde o {\code dpkg} esta fortemente ligado aos pacotes a serem instalados ou removidos, enquanto o {\code APT} esta relacionado aos repositórios onde são armazenados os pacotes. Já aplicações como o {\code aptitude} ou o {\code synaptic} estão voltadas a produzirem uma camada mais \textit{friendly-user}, com interfaces visuais e menos poluição textual.

Assim, definimos o termo \textit{gerenciadores de pacotes} àquelas aplicações dedicadas unicamente à instalação e remoção de pacotes, enquanto \textit{gerenciadores de repositórios} serve de denominação para as aplicações voltadas ao controle da lista de repositórios de pacotes e interface  para instalação, remoção e pesquisa de pacotes, fazendo de uso dos \textit{gerenciadores de pacotes} quando necessário.

Tendo em vista que este trabalho busca apresentar uma melhor solução para a ordenação de pacotes listados na resposta recebida em uma pesquisa realizada em um gerenciador de repositórios, devemos inicialmente elencar os possíveis candidatos a serem utilizados no trabalho.
% Sendo assim, os gerenciadores de repositórios que serão levados em consideração neste trabalho serão:

% TODO Transformar cada item da descrição em uma subsubsection, com dois ou
% três parágrafos (data de criação, autores, distros, principais vantagens/ 
% características, gerenciador de pacotes que roda por baixo (citações das documentações) %DONE

% TODO Levar o conteúdo desta seção para a subseção do apt
\subsubsection{APT} % (fold)
\label{subs:apt}

O {\code APT}, \textit{Advanced Package Tool}, é um gerenciador de repositórios  amplamente utilizado em distribuições oriundas da Debian (\textit{Debian-like}).
É provavelmente o gerenciador mais comum atualmente devido a popularidade de distribuições como \textbf{Ubuntu}, \textbf{Mint} e \textbf{Debian}.

Por ser um \textit{gerenciador de repositórios}, o {\code APT} trabalha unicamente com os arquivos \textit{.deb} que possuem seus respectivos endereços registrados no arquivo {\code/etc/apt/sources. list}, possibilitando o uso de diretórios remotos ou locais para a disponibilização dos pacotes a serem gerenciados. Requisitado a instalação de um pacote ao {\code APT}, será feito o \textit{download} do pacote e  suas respectivas dependências e repassado ao {\code dpkg} para que seja realizado o processo de instalação.

%(data de criação, autores, distros, principais vantagens/características, gerenciador de pacotes que roda por baixo (citações das documentações)

% Como introduzido em \ref{sub:visualiza_o_do_problema}, o \textit{APT} possui como repositório oficial o \url{git://anonscm.debian.org/apt/apt.git}, apesar de também ser hospedado em outros repositórios, tal como o \url{https://github.com/Debian/apt}. Uma observação mais atenta no \href{https://github.com/Debian/apt/blob/debian/experimental/cmdline/apt-cache.cc}{apt-cache.cc}, disponibilizado no repositório nos esclarece como o \textit{apt-cache} procede na ordenação dos pacotes para impressão.

%\begin{lstlisting}[language=C++]
%LocalitySort(&DFList->Df,Cache->HeaderP->GroupCount,sizeof(*DFList));
%\end{lstlisting}

\subsubsection{YUM} % (fold)
\label{ssub:yum}

O Yum é o gerenciador de repositórios padrão  das distribuições oriundas da \textbf{RedHat}, como o \textbf{Fedora} e \textbf{CentOS}.

\subsubsection{Emerge} % (fold)
\label{ssub:emerge}

O Emerge é comumente usado como o gerenciador de repositórios da distribuição \textbf{Gentoo} e suas ramificações. Normalmente o gerenciador é utilizado através do gerenciador de repositórios \textit{emerge}, este por sua vez escrito em \textit{Python} e \textit{Bash}.
% \subsubsection{Pacman} % (fold)
\label{ssub:pacman}

Dentre os gerenciadores de repositórios, o Pacman é provavelmente o mais recente dentre os mais usuais. Como não poderia deixar de ser, é  o gerenciador de repositórios de uma ramificação recente dentre distribuições Linux que vem conseguindo ganhar seu espaço entre distribuições mais tradicionais, a \textbf{Arch}, lançada em março de 2002

\subsubsection{Comandos básicos} % (fold)
\label{subs:comandos_basicos}

% TODO Escrever uma intro (3 a 4 parágrafos) descrevendo rapidamente os
% principais comandos (instalação, remoção, busca e consulta). Depois
% montar uma tabela de comandos por gerenciador %DONE

Por ser baseados em uma lista de repositórios, os \textit{gerenciadores de repositórios} recomendam uma atualização de seus bancos de dados antes de realizar qualquer sequencia de operações para garantir que seus \textit{links} serão validos. A Tabela \ref{sync_gerenciadores} apresenta os comandos para a atualização ou sincronização da lista de repositórios e pacotes dos gerenciadores. Os comandos de atualização de lista de repositórios precisam de permissão administrativa e devem ser relizados como usuário \textit{root} ou ter a invocação do comando {\code sudo} antes do comando de instalação/ remoção. Seguindo a padronização adotada pelas distribuições Linux para apresentação de comandos em terminal \cite{hekman1996linux}, comandos precedidos de \textbf{\code\#} implicam em comandos que necessitam de permissões administrativas enquanto comandos precedidos por \textbf{\code\$}  podem ser efetuados por qualquer usuário.

\begin{table}[htbp]
\caption{Comandos para atualização do banco de dados dos gerenciadores de repositórios mais populares.}
\centering
\begin{tabular}{|l|l|}
\hline
& \textbf{Atualizar banco de dados} \\ \hline
\textbf{\code APT} & {\code\# apt-get update}  \\ \hline
\textbf{\code YUM} & {\code\# yum update}  \\ \hline
\textbf{\code Emerge} & {\code\# emerge --sync}  \\ \hline
% \textbf{\code Pacman} & {\code\# pacman -Ss}  \\ \hline
\end{tabular}
\label{sync_gerenciadores}
\end{table}

Pressupondo que os repositórios já estejam sincronizados na maquina local, é possível realizar os comandos básicos de instalação, remoção ou busca por pacotes com a minimização dos riscos que um pacote não seja encontrado ou tenha sua referência incompleta ou invalida.  Estes comandos devem ser passados também via terminal. Os comandos de instalação e remoção de pacotes precisam de permissão administrativa e devem ser relizados como usuário \textit{root} ou ter a invocação do comando {\code sudo} antes do comando de instalação/ remoção. A Tabela \ref{cmd_gerenciadores} lista estes comandos para os gerenciadores de repositórios mais populares, onde {\code <pacote>} é o nome do pacote com o qual se deseja operar. É importante frisar que tem-se por padrão a nomeação dos pacotes sem caracteres especiais ou espaços. Apesar de haver uma padronização quanto ao versionamento dos pacotes e suas arquiteturas, não é necessário a caracterização destas informações. Para operações de instalação de pacotes, serão priorizados os pacotes mais recentes com a arquitetura compatível com a da maquina, sendo necessário a descriminação da arquitetura {\code x86} para a instalação de pacotes 32 \textit{bits} em sistemas 64 \textit{bits}.

\begin{table}[htbp]
\caption{Principais comandos dos gerenciadores de repositórios mais populares.}
\resizebox{\columnwidth}{!}{%
\begin{tabular}{|l|l|l|l|}
\hline
 & \multicolumn{1}{c|}{\textbf{Instalar}} & \multicolumn{1}{c|}{\textbf{Remover}} & \multicolumn{1}{c|}{\textbf{Procurar}} \\ \hline
\textbf{\code APT} & {\code\# apt-get install <pacote>} & {\code\# apt-get purge <pacote>} & {\code\$ apt-cache search <pacote>} \\ \hline
\textbf{\code YUM} & {\code\# yum install <pacote>} & {\code\# yum purge <pacote>} & {\code\$ yum search <pacote>} \\ \hline
\textbf{\code Emerge} & {\code\# emerge <pacote>} & {\code\# emerge remove <pacote>} & {\code\$ emerge search <pacote>} \\ \hline
% \textbf{\code Pacman} & {\code\# pacman -S <pacote>} & {\code\# pacman -Rc <pacote>} & {\code\$ pacman -Qi  <pacote>} \\ \hline
\end{tabular}
}
\label{cmd_gerenciadores}
\end{table}

