\chapter{\nmu Objetivos}
\label{cha:objetivos}

A primeiro momento, este trabalho tem como objetivo a apresentação de modelos distintos de ordenação no intuito de evidenciar a carência dos gerenciadores de pacotes atuais em apresentação de resultados de pesquisa por pacotes. Consciente do objetivo deste trabalho, será evitado o desenvolvimento dos códigos de algoritmo de ordenação, visto que o objetivo não é a implementação dos algoritmos, mas a comparação dos resultados obtidos através dos algoritmos de ordenação e suas performances nos estados atuais de implementação. Assim, serão utilizadas bibliotecas já implementadas dos algoritmos de ordenação, citando suas respectivas referências e créditos.

\section{Objetivo Geral}

Este trabalho tem como objetivo apresentar  métodos de busca que apresentem resultados mais próximos do esperado aos usuários, quando estes forem realizar a busca por pacotes disponíveis para instalação em suas distribuições.

\section{Objetivos Específicos}

Para o alcance de melhores resultados de busca, visa-se inicialmente apresentar alternativas de ordenação para  aplicações padrões de procura de pacotes nas distribuições. Devido ao tempo e equipe disponível para o desenvolvimento da proposta, espera-se restringir inicialmente o beneficio para a ferramenta que consiga atingir uma maior quantidade de usuários e, principalmente, usuário que estejam adentrando no universo Linux e ainda não dominam técnicas de controle de fluxo via terminal e assim são lesados quando tentam efetuar tarefas simples como a procura de um pacote por não terem conhecimento de realizar um filtro nos resultados obtidos, seja com ajuda de rotinas \textit{regex} (expressões regulares), sejam com paginações com o auxilio de comandos como {\code more} ou {\code less}.

%% Há a possibilidade de inserção do código de ordenação inicialmente  em {cmdline/apt-cache.cc}, influenciando diretamente o apt-cache. Há uma brecha para um patch em {apt-private/private-search} onde há um FIXME: SORT! na busca de ordenação alfabetica, por status, etc.
