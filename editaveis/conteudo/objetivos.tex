\chapter{\nmu Objetivos}
\label{cha:objetivos}

Primeiramente, este trabalho tem como objetivo a avaliação de diferentes algoritmos de \textit{string matching}, com o intuito de evidenciar o potencialde melhoria dos gerenciadores de pacotes atuais em relação à apresentação de resultados de pesquisa por pacotes. Consciente do objetivo deste trabalho, será evitado o desenvolvimento dos códigos de algoritmo de ordenação, visto que o objetivo não é a implementação dos algoritmos, mas a comparação dos resultados obtidos através dos algoritmos de \textit{string matching} e suas performances nos estados atuais de implementação. Assim, serão utilizadas bibliotecas que possuem implementações dos referidos algoritmos, citando suas respectivas referências e créditos.

\section{Objetivo Geral}

Este trabalho tem como objetivo apresentar  métodos de qualificação dos resultados de busca de modo que apresentem resultados mais próximos do esperado pelos usuários, quando estes forem realizar a busca por pacotes disponíveis para instalação em suas distribuições.

\section{Objetivos Específicos}

% Para o alcance de melhores resultados de busca, visa-se inicialmente apresentar alternativas de ordenação para  aplicações padrões de procura de pacotes nas distribuições. Devido ao tempo e equipe disponível para o desenvolvimento da proposta, espera-se restringir inicialmente o beneficio para a ferramenta que consiga atingir uma maior quantidade de usuários e, principalmente, usuário que estejam adentrando no universo Linux e ainda não dominam técnicas de controle de fluxo via terminal e assim são lesados quando tentam efetuar tarefas simples como a procura de um pacote por não terem conhecimento de realizar um filtro nos resultados obtidos, seja com ajuda de rotinas \textit{regex} (expressões regulares), sejam com paginações com o auxilio de comandos como {\code more} ou {\code less}.

%% Há a possibilidade de inserção do código de ordenação inicialmente  em {cmdline/apt-cache.cc}, influenciando diretamente o apt-cache. Há uma brecha para um patch em {apt-private/private-search} onde há um FIXME: SORT! na busca de ordenação alfabetica, por status, etc.

\begin{itemize}
	\item Estudo de algoritmos de \textit{string matching} que venham a qualificar a saída apresentada em uma busca por pacotes.
	\item Apresentar soluções que venham a ampliar a qualidade apresentada na busca por pacotes nos gerenciadores e tornando-os menos dependentes de ferramentas complementares como {\code grep}, {\code more} ou {\code less}.
	\item Apresentar modelos que possibilitem a apresentação de resultados de uma busca por pacotes, mesmo quando inserido o nome de um pacote inexistente ou com erros ortográficos.
	\item Reduzir a necessidade do uso de rotinas de \textit{regex} por parte do usuário para a localização de pacotes dentre uma lista de resultados obtidos pelo gerenciador.
\end{itemize}