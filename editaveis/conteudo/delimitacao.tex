\chapter{\nmu Delimitação do \nmu Assunto}
\label{cha:delimitacao}

Este trabalho visa apresentar alternativas de busca de pacotes que possam vir a facilitar  o processo de busca. Tendo em vista a proximidade dos resultados atualmente apresentados pelos gerenciadores de repositórios, o universo de estudo de gerenciadores será restringido para apenas um a fim de conseguir melhores opções de resultados de ordenação de pacotes e assim apresentar propostas que posam ser utilizadas nos distintos gerenciadores disponíveis, havendo a necessidade de desenvolvimento das interfaces para a utilização do algoritmo com as demais opções de gerenciadores disponíveis. Para tal, será usado como base de estudo o gerenciador \textit{APT}, devido a sua maior popularidade. A disponibilidade de  interfaces em \textit{Python} para o gerenciador é interessante também. Devido a simplicidade da linguagem para a prototipação de código, a disponibilidade de algoritmos de ordenação já existentes e simplicidade em controles com \textit{strings} faz da linguagem \textit{Python} uma escolha muito interessante para o desenvolvimento deste estudo.

Mesmo havendo um interesse de contribuição com a comunidade livre e evolução do sistema Linux para um estado mais próximo dos usuários leigos na plataforma, em primeiro ponto este trabalho não tem interesse em apresentar alternativas otimizadas e com grande performance. Tendo em vista haver um interesse em avaliação dos modelos de ordenação, é de se esperar obter resultados que necessitem de mais de $50$ segundos para a apresentação de resultados.

Outro ponto a se frisar é que, mesmo o \textit{APT} sendo hoje um conjunto de aplicações, ele é predominantemente escrito em linguagem C. A apresentação de resultados deste trabalho sendo escrita em \textit{Python} implica uma grande queda de performance se comparado com todo o resto da aplicação a qual o \textit{APT} representa.
