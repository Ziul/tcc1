\chapter{Introdução}
\label{cha:introducao}


% Este documento apresenta considerações gerais e preliminares relacionadas 
% à redação de relatórios de Projeto de Graduação da Faculdade UnB Gama 
% (FGA). São abordados os diferentes aspectos sobre a estrutura do trabalho, 
% uso de programas de auxilio a edição, tiragem de cópias, encadernação, etc.


\section{Visualização do problema} % (fold)
\label{sub:visualiza_o_do_problema}

Dentre os diversos gerenciadores de pacotes utilizados atualmente para a instalação de pacotes em distribuições Linux, podemos apontar alguns que se destacam por virem já instalados de padrão nas distribuições mais comuns hoje.

Porém, todos estes gerenciadores possuem um problema em comum. A apresentação de resultados de busca não é personalizável, não permitindo classificar a listagem de pacotes por nome por exemplo, nem fazem \textit{matching} de aproximação  entre o pacote pesquisado e os possíveis candidatos. Desta forma a procura por um pacote do qual não se tenha o nome correto pode se tornar incrivelmente cansativa e desgastante.
