\subsubsection{YUM} % (fold)
\label{ssub:yum}

O \textbf{Yellowdog Updater, Modified} (YUM) é o gerenciador de repositórios padrão  das distribuições oriundas da \textbf{Red Hat Linux}, como o \textbf{Fedora} e \textbf{CentOS} ou demais distribuições com sistema de pacotes baseados no padrão \textit{RPM}. Semelhante ao \textit{APT}, o \textit{YUM} consegue trabalhar com repositórios locais ou remotos com uma rede.
Semelhante a como o \textit{DPKG} é usado pelo \textit{APT} para o processo de instalação e remoção de pacotes, o \textit{YUM} faz de uso do 	\textit{RPM} como dependência para processamento de pacotes.

Escrito predominantemente em \textit{Python}, o \textit{YUM} surgiu  da reescrita do \textbf{Yellowdog Updater} (YUP) e tinha como objetivo a manutenção do gerenciador para sistemas Red Hat Linux no departamento de física da Universidade de Duke e teve como seus desenvolvedores iniciais Seth Vidal e Michael Stenner \cite{yum-howto}, tendo sua primeira documentação publicada em 2003 por Robert G. Brown \cite{yum_article}, foi reconhecida como o gerenciador de repositórios padrão para sistemas \textit{RPM-like} apenas em 2007 \cite{fusco2007linux}.

Por ser escrita em \textit{Python}, o YUM apresenta uma performance de desempenho inferior ao APT (escrito predominantemente em C++), porém possui uma interface com o usuário em linha de comando muito mais amistosa e intuitiva.