\definecolor{blue}{RGB}{41,5,195}
\makeatletter
\hypersetup{
     	%pagebackref=true,
		pdftitle={\@title}, 
		pdfauthor={\@author},
    	pdfsubject={\imprimirpreambulo},
	    pdfcreator={LaTeX with abnTeX2},
		pdfkeywords={abnt}{latex}{abntex}{abntex2}{trabalho acadêmico}, 
		colorlinks=true,       		% false: boxed links; true: colored links
    	linkcolor=blue,          	% color of internal links
    	citecolor=blue,        		% color of links to bibliography
    	filecolor=magenta,      		% color of file links
		urlcolor=blue,
		bookmarksdepth=4
}
\makeatother
\setlength{\parindent}{1.3cm}
\setlength{\parskip}{0.2cm}  
\makeindex


%--------------------------------------------
%Definições para código com fundo listrado
\newcommand\realnumberstyle[1]{#1}
\makeatletter
\newcommand{\zebra}[3]{%
    {\realnumberstyle{#3}}%
    \begingroup
    \lst@basicstyle
    \ifodd\value{lstnumber}%
        \color{#1}%
    \else
        \color{#2}%
    \fi
        \rlap{
        \color@block{0.94\textwidth}{\ht\strutbox}{\dp\strutbox}%
        \hspace*{\lst@numbersep}%
        }%
    \endgroup
}
\makeatother
\renewcommand\lstlistingname{Código}

\DeclareCaptionFormat{listing}{\vskip1pt\rule{\dimexpr\textwidth\relax}{0.4pt}\par\vskip1pt{\centering#1#2$-$ #3}}
\captionsetup[lstlisting]{format=listing,singlelinecheck=false, margin=0pt, font={sf},labelsep=space,labelfont=bf}

\DeclareMathVersion{sans}
\SetSymbolFont{operators}{sans}{OT1}{cmbr}{m}{n}
\SetSymbolFont{letters}  {sans}{OML}{cmbrm}{m}{it}
\SetSymbolFont{symbols}  {sans}{OMS}{cmbrs}{m}{n}

\lstnewenvironment{sflisting}[1][]
  {\lstset{#1}\mathversion{sans}}{}

\lstset{%
language=C++,           %linguagem
numbers=left,           %posição dos números
stepnumber=1,           %frequencia de aparição dos números
numbersep=5pt,
numberstyle=\zebra{gray!15}{white!35},
basewidth={0.6em,0.45em},
fontadjust=true,
mathescape=true,
tabsize=4,
commentstyle=\color{blue},
literate={á}{{\'a}}1 {à}{{\`a}}1 {ã}{{\~a}}1 {é}{{\'e}}1 {É}{{\'E}}1 {ê}{{\^e}}1 {õ}{{\~o}}1 {í}{{\'i}}1 {ó}{{\'o}}1 {ú}{{\'u}}1 {ç}{{\c c}}1 {³}{{$^3$}}1 {Ω}{{$\Omega$}}1,
breaklines=true,
showstringspaces=false,
stringstyle=\color{cyan},
basicstyle=\small\ttfamily}

\lstset{%
language=Python,           %linguagem
numbers=left,           %posição dos números
stepnumber=1,           %frequencia de aparição dos números
numbersep=5pt,
numberstyle=\zebra{gray!15}{white!35},
basewidth={0.6em,0.45em},
fontadjust=true,
mathescape=true,
tabsize=4,
commentstyle=\color{blue},
literate={á}{{\'a}}1 {à}{{\`a}}1 {ã}{{\~a}}1 {é}{{\'e}}1 {É}{{\'E}}1 {ê}{{\^e}}1 {õ}{{\~o}}1 {í}{{\'i}}1 {ó}{{\'o}}1 {ú}{{\'u}}1 {ç}{{\c c}}1 {Ω}{{$\Omega$}}1,
breaklines=true,
showstringspaces=false,
% inputencoding=latin1,
stringstyle=\color{cyan},
xleftmargin=\leftmargini,
columns=flexible,
basicstyle=\small\ttfamily}

%% Tabelas
%------------------------------

\DeclareCaptionSubType*[fnsymbol]{table}
\newcommand{\code}{\small\ttfamily}