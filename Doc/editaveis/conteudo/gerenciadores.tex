\section{Gerenciadores de pacotes} % (fold)
\label{sec:distribui_es_abordadas}

\subsection{Contextualização} % (fold)
\label{sec:breve_descri_o}

Desde o lançamento das distribuições \textit{Debian} e \textit{Slackware} em 1993, diversas distribuições tiveram seu ciclo de vida realizado por completo, apresentando e amadurecendo a filosofia por trás da distribuição, gerando novas ramificações e por fim sendo abandonadas. Hoje, se escolhermos alguma distribuição Linux dentre as inúmeras existentes, provavelmente ela será descendente de uma das principais distribuições: \textit{Debian, Red Hat} ou \textit{Slackware}. % ou \textit{Arch}, 
A \autoref{fig:figuras_linux_timeline_heranca}, no Anexo \ref{anexos} apresenta um modelo de herança entre as distribuições Linux onde pode-se observar a influência que estas distribuições tiveram em relação a um grupo de novas distribuições que se ramificaram delas%
\footnote{Apesar da difícil visualização da imagem no documento, o intuito dela aqui é apresentar a vasta quantidade de distribuições que surgiram de bifurcações das distribuições \textit{Debian, Red Hat} e \textit{Slackware}, tal como alterações nelas provavelmente viriam a influenciar as demais distribuições.}.

Em sistemas operacionais Linux há muito foi proposto uma alternativa para distribuições de aplicações para o sistemas: os pacotes. A filosofia por trás da comunidade livre \cite{bretthauer2001open} e o interesse em buscar novas formas de armazenamento e distribuição de pacotes fez com que surgissem diversos \textit{gerenciadores de pacotes}, voltados à determinadas distribuição  cada \cite{beck2002linux}.

Devido grande parte das distribuições hoje existentes serem ramificações das distribuições \textit{Debian, Red Hat} ou \textit{Slackware},% ou \textit{Arch}, 
os diversos gerenciadores de pacotes hoje disponíveis também possuem muitas similaridades entre eles.
É o caso do \textit{dpkg} e o \textit{rpm}, onde as funcionalidades de instalação, remoção e provimento de informações de pacotes esta presente em ambos. Ambos também são escritos predominantemente em \textit{C} e \textit{Perl}. Suas diferenças surgem quando chegamos a averiguar quais pacotes cada aplicação suporta. O \textit{dpkg} esta voltado para os pacotes \textit{.deb}, enquanto o \textit{rpm} esta voltado para os pacotes \textit{rpm}. Como há uma distinção em como estes pacotes são montados \cite{bailey1997maximum}, há uma incompatibilidade entre as duas aplicações. 

% TODO: Acrescentar um parágrafo citando alguns gerenciadores de pacotes e formatos (dpkg/deg, rpm, etc). %DONE

\subsection{Gerenciadores de pacotes e Gerenciadores de Repositórios} % (fold)
\label{sec:gerenciadores}

% TODO Definir e diferenciar gerenciadores de pacotes e gerenciadores de
% repositórios % DONE

Um ponto de vista importante de ser caracterizado neste trabalho é a diferença entre os termos \textit{ gerenciadores de pacotes} e \textit{gerenciadores de repositórios}.
É importante caracterizar a diferença estes dois termos comumente utilizados como sinônimos devido a sua similaridade. A proximidade é tamanha que na própria documentação oficial do \textit{Debian} não há uma caracterização de \textit{gerenciadores de repositórios}, citando apenas que gerenciadores como o {\code aptitude} ou {\code dselect} dependem do {\code APT}, que por sua vez depende do {\code dpkg} \cite{debian-faq}. Essa cadeia de dependências indicam as camadas de interface, onde o {\code dpkg} é responsavel pela instalação, remoção e gerência dos pacotes, enquanto o {\code APT} está relacionado aos repositórios onde são armazenados os pacotes. Já aplicações como o {\code aptitude} ou o {\code synaptic} visam oferecer uma camada \textit{user friendly}, com interfaces gráficas e menos informações textuais.

Assim, definimos o termo \textit{gerenciadores de pacotes} àquelas aplicações dedicadas ao manuseio dos pacotes( instalação e remoção, por exemplo), enquanto \textit{gerenciadores de repositórios} são as aplicações voltadas ao controle da lista de repositórios de pacotes e interface  para instalação, remoção e pesquisa de pacotes, fazendo de uso dos \textit{gerenciadores de pacotes} quando necessário.

Tendo em vista que este trabalho busca apresentar uma melhor apresentação dos resultados de uma busca por um pacote a ser instalado em uma pesquisa realizada em um gerenciador de repositórios, devemos inicialmente elencar os possíveis candidatos a serem avaliados no trabalho.
% Sendo assim, os gerenciadores de repositórios que serão levados em consideração neste trabalho serão:

% TODO Transformar cada item da descrição em uma subsubsection, com dois ou
% três parágrafos (data de criação, autores, distros, principais vantagens/ 
% características, gerenciador de pacotes que roda por baixo (citações das documentações) %DONE

% TODO Levar o conteúdo desta seção para a subseção do apt
\subsubsection{APT} % (fold)
\label{subs:apt}

O {\code APT}, \textit{Advanced Package Tool}, é um gerenciador de repositórios  amplamente utilizado em distribuições oriundas da Debian (\textit{Debian-like}).
É provavelmente o gerenciador mais comum atualmente devido a popularidade de distribuições como \textbf{Ubuntu}, \textbf{Mint} e \textbf{Debian}.
Projetado inicialmente para uma substituição do gerenciador {\code dselect}, o APT teve suas primeiras \textit{builds} distribuídas via IRC em Agosto de 1998,  sendo integrado ao \textit{Debian} na \textit{release} de Março de 1999 \cite{garbee2008brief}. O  APT pode ser considerado como uma interface para o {\code dpkg}, gerenciando as dependências de um pacote para a instalação e remoção, alem de apresentar uma listagem dos pacotes disponíveis na lista de repositórios. Um dos recursos que o APT realiza de forma transparente para o usuário é a ordenação de pacotes para instalação ou remoção com o uso das chamadas do {\code dpkg}, suprindo as dependências dos pacotes que são requisitos para o pacote ao qual o usuário deseja instalar na maquina. Dentre as desvantagens que o APT pode apresentar, as que se destacam são justamente o fato de estar escrito em C++, que apesar do ganho de performance que apresenta aos seus concorrentes, carrega junto uma maior dificuldade de manutenção devido ao tamanho e complexidade. 

Outro problema do APT é sua fragmentação. Apesar de serem tratados como parte do APT, as aplicações como {\code apt-get} ou {\code apt-cache} são na realidade aplicações que fazem de uso do APT como uma biblioteca para interface, mas são comumente tratadas como aplicações integradas ao APT. Como consequências, o APT possuiu uma das interfaces menos intuitivas quando comparado com gerenciadores como o \textit{YUM} ou o \textit{Portage} justamente por frequentemente serem apresentadas soluções utilizando as aplicações como o {\code apt-get} ao invés do {\code apt}.

Por ser um \textit{gerenciador de repositórios}, o {\code APT} trabalha unicamente com os arquivos \textit{.deb} que possuem seus respectivos endereços registrados no arquivo {\code/etc/apt/sources. list}, possibilitando o uso de diretórios remotos ou locais para a disponibilização dos pacotes a serem gerenciados. Requisitado a instalação de um pacote ao {\code APT}, será feito o \textit{download} do pacote e  suas respectivas dependências e repassado ao {\code dpkg} para que seja realizado o processo de instalação.

O \textit{APT} possui como repositório oficial o \url{git://anonscm.debian.org/apt/apt.git}, apesar de também ser hospedado em outros repositórios, tal como o \url{https://github.com/Debian/apt}. Neste trabalho estaremos trabalhando em busca de soluções que venha a contribuir e enriquecer em especial a aplicação \href{https://github.com/Debian/apt/blob/debian/experimental/cmdline/apt-cache.cc}{apt-cache}, em especial o parâmetro \textbf{search}. Segundo podemos observar o código disponibilizado do \textit{APT}, sua apresentação de resultados é ordenada alfabeticamente de acordo com a lista de repositórios disponíveis.


%(data de criação, autores, distros, principais vantagens/características, gerenciador de pacotes que roda por baixo (citações das documentações)


%\begin{lstlisting}[language=C++]
%LocalitySort(&DFList->Df,Cache->HeaderP->GroupCount,sizeof(*DFList));
%\end{lstlisting}

\subsubsection{YUM} % (fold)
\label{ssub:yum}

O Yum é o gerenciador de repositórios padrão  das distribuições oriundas da \textbf{RedHat}, como o \textbf{Fedora} e \textbf{CentOS}.

\subsubsection{Portage} % (fold)
\label{ssub:emerge}

O Portage é  o gerenciador de repositórios padrão da distribuição \textbf{Gentoo} e suas ramificações \cite{vermeulen2005gentoo}. O Portage é na verdade duas aplicações trabalhando em conjunto para o funcionamento de um todo: o {\code ebuild} que cuida do processo de \textit{building} e instalação dos pacotes enquanto o {\code emerge} providência a interface ao usuário. O Portage foi predominantemente escrito em \textit{Python}.

O Portage, por padrão, não faz de uso de pacotes binários, tais como RPM ou DEB, mas sim a compilação da aplicação durante o processo de instalação. Com resultados o Portage mantem uma alta compatibilidade da aplicação com a maquina e alta flexibilidade com as arquiteturas de suporte, ao custo da demora para a instalação.
Essa flexibilidade permite que o Portage não seja em momento algum dependente do sistema, sendo por vezes utilizado para gerenciar outros sistemas, tais como o BSD, Mac OS e Solaris \cite{ryan2010linux}.
Eventualmente, \textit{forks} do Portage são por vezes apresentados como soluções para sistemas embarcados pelo fato da portabilidade da aplicação depender apenas da presença do compilador \cite{guointegrated}. Todavia, o Portage apresenta também uma solução de instalação com empacotamento de binários através das \textit{flags} {\code --usepkg --getbinpkg} \cite{gentoo-doc}  para instalações mais rápidas.
% \subsubsection{Pacman} % (fold)
\label{ssub:pacman}

Dentre os gerenciadores de repositórios, o Pacman é provavelmente o mais recente dentre os mais usuais. Como não poderia deixar de ser, é  o gerenciador de repositórios de uma ramificação recente dentre distribuições Linux que vem conseguindo ganhar seu espaço entre distribuições mais tradicionais, a \textbf{Arch}, lançada em março de 2002

\subsubsection{Comandos básicos} % (fold)
\label{subs:comandos_basicos}

% TODO Escrever uma intro (3 a 4 parágrafos) descrevendo rapidamente os
% principais comandos (instalação, remoção, busca e consulta). Depois
% montar uma tabela de comandos por gerenciador %DONE

Por ser baseados em uma lista de repositórios, os \textit{gerenciadores de repositórios} recomendam uma atualização de seus bancos de dados antes de realizar qualquer sequência de operações para garantir que seus \textit{links} serão válidos. A \autoref{sync_gerenciadores} apresenta os comandos para a atualização ou sincronização da lista de repositórios e pacotes dos gerenciadores. Os comandos de atualização de lista de repositórios precisam de permissão administrativa e devem ser relizados como usuário \textit{root} ou ter a invocação do comando {\code sudo} antes do comando de instalação/ remoção. Seguindo a padronização adotada pelas distribuições Linux para apresentação de comandos em terminal \cite{hekman1996linux}, comandos precedidos de \textbf{\code\#} implicam em comandos que necessitam de permissões administrativas enquanto comandos precedidos por \textbf{\code\$}  podem ser efetuados por qualquer usuário.

\begin{table}[htbp]
\caption{Comandos para atualização do banco de dados dos gerenciadores de repositórios mais populares.}
\centering
\begin{tabular}{ll}
\toprule
\textbf{Gerenciador de Repositórios} & \textbf{Commando} \\ 
\midrule
\textbf{\code APT} & {\code\# apt-get update}  \\ 
\rowcolor[gray]{0.9}
\textbf{\code YUM} & {\code\# yum update}  \\ 
\textbf{\code Emerge} & {\code\# emerge --sync}  \\ 
\bottomrule
% \textbf{\code Pacman} & {\code\# pacman -Ss}  \\ \hline
\end{tabular}
\label{sync_gerenciadores}
\end{table}

Pressupondo que os repositórios já estejam sincronizados na maquina local, é possível realizar os comandos básicos de instalação, remoção ou busca por pacotes sem riscos de que um pacote não seja encontrado ou tenha sua referência incompleta ou inválida.  Estes comandos devem ser passados também via terminal. Os comandos de instalação e remoção de pacotes precisam de permissão administrativa. A \autoref{cmd_gerenciadores} lista estes comandos para os gerenciadores de repositórios mais populares, onde {\code <pacote>} é o nome do pacote com o qual se deseja operar. É importante frisar que tem-se por padrão a nomeação dos pacotes sem caracteres especiais ou espaços. Apesar de existir uma padronização quanto ao versionamento dos pacotes e suas arquiteturas, não é necessário a caracterização destas informações. Para operações de instalação de pacotes, serão priorizados os pacotes mais recentes com arquitetura compatível com a máquina, sendo necessário a descriminação da arquitetura {\code x86} para a instalação de pacotes 32 \textit{bits} em sistemas 64 \textit{bits}.

\begin{table}[htbp]
\caption{Principais comandos dos gerenciadores de repositórios mais populares.}
\resizebox{\columnwidth}{!}{%
\begin{tabular}{llll}
\toprule
 \multicolumn{1}{c}{\textbf{Gerenciador}} & \multicolumn{1}{c}{\textbf{Instalar}} & \multicolumn{1}{c}{\textbf{Remover}} & \multicolumn{1}{c}{\textbf{Procurar}} \\ \midrule
\textbf{\code APT} & {\code\# apt-get install <pacote>} & {\code\# apt-get purge <pacote>} & {\code\$ apt-cache search <pacote>} \\
\rowcolor[gray]{0.9}
\textbf{\code YUM} & {\code\# yum install <pacote>} & {\code\# yum purge <pacote>} & {\code\$ yum search <pacote>} \\ 

\textbf{\code Emerge} & {\code\# emerge <pacote>} & {\code\# emerge remove <pacote>} & {\code\$ emerge search <pacote>} \\ \bottomrule
% \textbf{\code Pacman} & {\code\# pacman -S <pacote>} & {\code\# pacman -Rc <pacote>} & {\code\$ pacman -Qi  <pacote>} \\ \hline
\end{tabular}
}
\label{cmd_gerenciadores}
\end{table}

