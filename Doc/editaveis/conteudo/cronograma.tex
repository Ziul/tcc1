\chapter{Cronograma}
\label{cha:cronograma}

% \section{Estudo atual} % (fold)
% \label{sec:estudo_atual}

Abaixo segue o gráfico de \textit{gantt} de evolução do trabalho contabilizadas em semanas, com evolução de tarefas estimadas, tendo a primeira semana em 21 de Agosto de 2013.

\begin{ganttchart}[
	%canvas/.style={fill=yellow!25,draw=blue, dashed, very thick}
	hgrid=true,
	vgrid={*1{blue, dotted}},
	%vgrid={*2{red}, *1{green}, *{10}{blue, dashed}}
	x unit = 8mm,
	today=47,
	today label= 17 de novembro,
	today label font=\itshape\color{red},
	today rule/.style={draw=red, ultra thick, dashed},
]{35}{48}%{2014-08-28}{2013-08-13}}
\gantttitle{Semanas}{14} \\
\gantttitlelist{1,...,14}{1}\\
%\gantttitle{}{13} \\
%
\ganttgroup{Fundamentação}{35}{40} \\
\ganttbar{1}{35}{36} \\ % Levantar documentação oficial das distribuições
\ganttbar{2}{35}{36} \\ % Levantar gerenciadores das principais distribuições Linux
\ganttbar{3}{35}{36} \\ % Configurar maquinas Vagrant
\ganttbar{4}{35}{38} \\ % Estudar documentação de String Matching
%
\ganttgroup{Prototipagem}{40}{43} \\
\ganttbar{5}{40}{42} \\ % Escrever primeiro protótipo
\ganttbar{6}{40}{42} \\ % Testar com 5 pacotes
\ganttbar{7}{41}{43} \\ % Escrever segundo protótipo
\ganttbar{8}{42}{43} \\ % Escrever terceiro protótipo
%
\ganttgroup{Documentação}{38}{47} \\
\ganttbar{9}{38}{40} \\ % Estruturar documento do TCC
\ganttbar{10}{38}{40} \\ % Escrever fundamentação teórica
\ganttbar{11}{43}{45} \\ % Escrita do TCC
\ganttbar{12}{45}{47} \\ % Revisão do TCC
\ganttbar{13}{46}{47} \\ % Produção da Apresentação
\ganttbar{14}{47}{47} \\ % Entrega
\end{ganttchart}

\textbf{Legenda dos índices do gráfico:}
\begin{description}
\item [1]  Levantar documentação oficial das distribuições.
\item [2]  Levantar gerenciadores das principais distribuições Linux.
\item [3]  Configurar máquinas com o \textit{Vagrant}.
\item [4]  Estudar documentação de \textit{String Matching}.
\item [5]  Escrever primeiro protótipo.
\item [6]  Testar com 5 pacotes.
\item [7]  Escrever segundo protótipo.
\item [8]  Escrever terceiro protótipo.
\item [9]  Estruturar documento do Trabalho de Conclusão de Curso.
\item [10] Escrever fundamentação teórica.
\item [11] Escrita do Trabalho de Conclusão de Curso.
\item [12] Revisão do Trabalho de Conclusão de Curso.
\item [13] Planejamento da Apresentação.
\item [14] Entrega.
\end{description}

Este trabalho foi planejado com entregas semanais com três grandes frentes: estudo da fundamentação teórica, prototipagem e documentação. O gráfico de \textit{gantt} a seguir apresenta uma aproximação do desenvolvimento deste trabalho, tendo ocorrido algumas adaptações durante o período da fundamentação\footnote{Até o desenvolvimento desta seção, o período de entrega ainda não havia ocorrido, fazendo com que a parte final do gráfico ainda seja estimada.}.

\section{Planejamentos futuros} % (fold)
\label{sec:planejamentos_futuros}

Para a continuação do estudo, algumas atividades já estão planejadas. São elas:

\begin{enumerate}
	\item Fork do repositório APT e configurações de ambiente necessárias.
	\item Revisão e estudo do código APT.
	\item Busca por bibliotecas contendo o algoritmo escolhido implementadas em C/C++\footnote{Preferência por bibliotecas em C++ por melhor compatibilidade e mais opções de  funcionalidades ou recursos}.
	\item Tradução do código Python para C++\footnote{Existe a possibilidade de se considerar outras linguagens além do C++ para implementação final, porém devido o APT estar em C++, uma tradução para esta linguagem proporcionaria uma melhor compatibilidade, oferecendo talvez tempo para a implementação de recursos sobressalentes para a aplicação.}.
	\item Fusão do algoritmo ao APT.
\end{enumerate}

Implementadas estas atividades, o objetivo do trabalho já estaria alcançado, porém a possibilidade de sobra de tempo permitiria que algumas funcionalidades pudessem vir a serem estudadas e planejadas para a implementação, tais como:

\begin{description}
	\item [Novas opções de saída/ ordenação dos pacotes] Pacotes poderiam ser ordenados alfabeticamente, por proximidade com a \textit{string} de entrada, por ordem de apresentação no repositório ou outras, sendo esta ordenação ser escolhida por \textit{flags} passadas durante a pesquisa.
	\item [Otimização do algoritmo] Um estudo para que o algoritmo fizesse uso de \textit{threads} ou multiprocessos para obter um ganho de performance. Talvez um estudo de tecnologias ainda mais alternativas como o uso de \textit{frameworks} como CUDA ou OpenCL para fazer de uso de recursos de GPU para ganho de desempenho.
	\item [Inclusão de possibilidade de prefixos e/ ou sufixos] A possibilidade de escolher com \textit{flags} o uso de prefixos ou sufixos para um melhor refinamento e classificação dos pacotes de maior interesse.
\end{description}

Devido ao resultado obtido na primeira parte do estudo, a adoção de uma média de duas semanas no máximo para cada atividade viria a ser uma escolha interessante para a segunda etapa do estudo, porém seriam necessários ao menos dez semanas para a obtenção dos resultados mínimos desejados para objetivo e até dezesseis semanas para a elaboração com todos os recursos desejados. É de interesse que as alterações sejam realmente adotadas pela comunidade, de forma que a consideração de ao menos uma semana para que a equipe do APT pudesse estudar o \textit{pull request} gerado pelo estudo deve ser considerado no cálculo do tempo disponível.


% chapter planejamentos_futuros (end)