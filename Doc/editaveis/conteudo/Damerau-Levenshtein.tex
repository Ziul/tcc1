\subsection{Damerau-Levenshtein} % (fold)
\label{sec:damerau_levenshtein}

A distância de Damerau-Levenshtein é uma variação da distância de Levenshtein \cite{levenshtein1965}, havendo ainda a adição da transposição no \textit{set} de operações básicas. Em seus estudos, Damerau apresentou que mais de $80\%$ dos erros de escrita oriundo de humanos estavam relacionados às operações de falta de caractere, excesso de caracteres, substituição de caracteres ou transposição de caracteres \cite{damerau1964technique}.


Por  considerar  a transposição de dois caracteres adjacentes, a distância de \textit{Damerau-Levenshtein} abriu portas para um novo padrão de métricas, as biológicas, no campo de mensurar a variação entre moléculas de DNA. Esta seria a brecha necessária para que surgissem outros métodos de analise de moléculas de DNA baseadas na comparação de \textit{strings}, tais como o algoritmo de \textit{Needleman–Wunsch} \cite{needleman1970general} e  o de \textit{Smith-Waterman} \cite{smith1981identification}, apresentado  na Seção \ref{sec:smith_waterman}.

\subsubsection{Implementação} % (fold)
\label{sub:implementa_damerau_levenshtein}

Para melhor contextualização da diferença entre o método tradicional de cálculo de distância e o modelo de \textit{Damerau-Levenshtein}, consideremos a seguir uma visão atômica de como seria o procedimento do cálculo da distância de Levenshtein, conforme mostrado no \autoref{levenshtein_distance_py}.

\begin{lstlisting}[language=Python,label=levenshtein_distance_py,caption={Visão atômica do cálculo da distância de Levenshtein}]
def levenshtein_distance(s1, s2):
    lenstr1 = len(s1)
    lenstr2 = len(s2)
    d = initialize_distance(lenstr1,lenstr2)

	d = calculate_distance(lenstr1,lenstr2,d) 

    return d[lenstr1-1,lenstr2-1]
\end{lstlisting}

O \autoref{levenshtein_distance_py} serviria tanto para o modelo tradicional como para o de \textit{Damerau-Levenshtein}, necessitando apenas implementar o método {\code calculate\_distance()} de forma distinta. O \autoref{damerau_levenshtein_distance_py} representa como seria a implementação para o modelo de \textit{Damerau-Levenshtein}.

\begin{lstlisting}[language=Python,label=damerau_levenshtein_distance_py,caption={Implementação da distância de Damerau-Levenshtein}]
def calculate_distance(lenstr1,lenstr2,d):
    for i in xrange(lenstr1):
        for j in xrange(lenstr2):
            if s1[i] == s2[j]:
                cost = 0
            else:
                cost = 1
            d[(i,j)] = min(
                           d[(i-1,j)] + 1, # deletion
                           d[(i,j-1)] + 1, # insertion
                           d[(i-1,j-1)] + cost, # substitution
                          )
            if i and j and s1[i]==s2[j-1] and s1[i-1] == s2[j]:
                d[(i,j)] = min (d[(i,j)], d[i-2,j-2] + cost) # transposition
 
    return d
\end{lstlisting}

A distinção do método de \textit{Damerau-Levenshtein} para o modelo tradicional é apenas a inserção das linhas \textbf{13} e \textbf{14} do \autoref{damerau_levenshtein_distance_py}, que realiza a transposição dos caracteres, quando for o caso, ao custo de uma única operação.

% subsection implementa_o_do_damerau_levenshtein (end)
%http://en.wikipedia.org/wiki/Damerau%E2%80%93Levenshtein_distance
% section damerau_levenshtein (end)
