\chapter{\nmu Delimitação do \nmu Assunto}
\label{cha:delimitacao}

Este trabalho visa apresentar métodos de qualificação dos resultados de busca por pacotes de repositórios. Tendo em vista a proximidade dos resultados atualmente apresentados pelos gerenciadores de repositórios, o universo de estudo de gerenciadores será restringido para apenas um, a fim de manter o foco na busca de melhores qualificações e assim apresentar propostas que possam ser utilizadas nos distintos gerenciadores disponíveis.

Para tal, será usado como base de estudo o gerenciador \textit{APT}, devido a sua maior popularidade. A disponibilidade de  interfaces em \textit{Python} para o gerenciador é interessante também. Devido a simplicidade da linguagem para a prototipação de código, a disponibilidade de algoritmos de comparação de \textit{strings} já existentes e simplicidade na manipulação de \textit{strings} faz da linguagem \textit{Python} uma escolha interessante para o desenvolvimento deste estudo.

Mesmo havendo um interesse em contribuir com a comunidade livre e com a evolução do sistema Linux para um estado mais próximo dos usuários leigos da plataforma, no primeiro momento o foco do trabalho não é apresentar alternativas otimizadas e com grande performance. Tendo em vista haver um interesse em avaliação dos algoritmos de comparação de \textit{strings}, os primeiros protótipos apresentam soluções que podem levar mais de $50$ segundos para a apresentação de resultados.

Outro ponto a se frisar é que, mesmo o \textit{APT} sendo hoje um conjunto de aplicações, ele é predominantemente escrito em linguagem C++. A apresentação de resultados deste trabalho, sendo escrita em \textit{Python}, apresenta uma performance inferior se comparada com todo o resto da aplicação a qual o \textit{APT} representa. %Além de estarem hoje sendo estudada a aplicação {\code apt-cache}, esta nada mais é que uma interface do \textit{APT} como saída otimizada para o comando {\code apt search} sem apresentar nenhuma ordenação além da alfabética, já inicialmente realizada pelo repositório.
