\subsubsection{Portage} % (fold)
\label{ssub:emerge}

O Portage é  o gerenciador de repositórios padrão da distribuição \textbf{Gentoo} e suas ramificações \cite{vermeulen2005gentoo}. O Portage consiste, de fato, em duas aplicações trabalhando em conjunto para o funcionamento de um todo: o {\code ebuild} cuida do processo de \textit{building} e instalação dos pacotes enquanto o {\code emerge} providência a interface ao usuário. O Portage foi predominantemente escrito em \textit{Python}.

O Portage, por padrão, não faz de uso de pacotes binários, tais como RPM ou DEB, mas sim códigos-fontes que são compilados durante o processo de instalação. Como resultado o Portage mantém uma alta compatibilidade da aplicação com a máquina e alta flexibilidade com as arquiteturas de suporte, ao custo da demora na instalação.
Essa flexibilidade permite que o Portage não seja em momento algum dependente do sistema, sendo por vezes utilizado para gerenciar outros sistemas, tais como o BSD, Mac OS e Solaris \cite{ryan2010linux}.
Eventualmente, \textit{forks} do Portage são por vezes apresentados como soluções para sistemas embarcados pelo fato da portabilidade da aplicação depender apenas da presença do compilador \cite{guointegrated}. Todavia, o Portage apresenta também uma solução de instalação com empacotamento de binários através das \textit{flags} {\code --usepkg --getbinpkg} \cite{gentoo-doc}  para instalações mais rápidas.