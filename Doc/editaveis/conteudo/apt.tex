% TODO Levar o conteúdo desta seção para a subseção do apt
\subsubsection{APT} % (fold)
\label{subs:apt}

O {\code APT}, \textit{Advanced Package Tool}, é um gerenciador de repositórios  amplamente utilizado em distribuições oriundas da Debian (\textit{Debian-like}).
É provavelmente o gerenciador mais comum atualmente devido a popularidade de distribuições como \textbf{Ubuntu}, \textbf{Mint} e \textbf{Debian}.
Projetado inicialmente para substituir o gerenciador {\code dselect}, o APT teve suas primeiras \textit{builds} distribuídas via IRC em Agosto de 1998,  sendo integrado ao \textit{Debian} na \textit{release} de Março de 1999 \cite{garbee2008brief}. O  APT pode ser considerado como uma interface para o {\code dpkg}, gerenciando as dependências de um pacote para a instalação e remoção, alem de apresentar uma listagem dos pacotes disponíveis na lista de repositórios. 

Um dos recursos que o APT fornece de forma transparente para o usuário é a ordenação de pacotes para instalação ou remoção com o uso das chamadas do {\code dpkg}, suprindo as dependências dos pacotes que são requisitos para o pacote ao qual o usuário deseja instalar na máquina. Dentre as desvantagens que o APT pode apresentar, as que se destacam são justamente o fato de estar escrito em C++, que apesar do ganho de performance que apresenta aos seus concorrentes, carrega junto uma maior dificuldade de manutenção devido ao tamanho e complexidade. 

Outro problema do APT é sua fragmentação. Apesar de serem tratados como parte do APT, as aplicações como {\code apt-get} ou {\code apt-cache} são na realidade aplicações que fazem de uso do APT como uma biblioteca para interface, mas são comumente tratadas como aplicações integradas ao APT. Como consequências, o APT possuiu uma das interfaces menos intuitivas quando comparado com gerenciadores como o \textit{YUM} ou o \textit{Portage} justamente por frequentemente serem apresentadas soluções utilizando as aplicações como o {\code apt-get} ao invés do {\code apt}.

Por ser um \textit{gerenciador de repositórios}, o {\code APT} trabalha unicamente com os arquivos \textit{.deb} que possuem seus respectivos endereços registrados no arquivo {\code/etc/apt/sources. list}, possibilitando o uso de diretórios remotos ou locais para a disponibilização dos pacotes a serem gerenciados. Requisitada a instalação de um pacote ao {\code APT}, será feito o \textit{download} do pacote e  suas respectivas dependências e estes arquivos são repassados ao {\code dpkg} para que seja realizado o processo de instalação.

O \textit{APT} possui como repositório oficial o \url{git://anonscm.debian.org/apt/apt.git}, apesar de também ser hospedado em outros repositórios, tal como o \url{https://github.com/Debian/apt}. Neste trabalho estaremos trabalhando em busca de soluções que venha a contribuir e enriquecer em especial a aplicação \href{https://github.com/Debian/apt/blob/debian/experimental/cmdline/apt-cache.cc}{apt-cache}, em especial o comando \textbf{search}. Segundo podemos observar o código disponibilizado do \textit{APT}, a apresentação dos resultados de uma busca são ordenados alfabeticamente, de acordo com a lista de repositórios disponíveis.


%(data de criação, autores, distros, principais vantagens/características, gerenciador de pacotes que roda por baixo (citações das documentações)


%\begin{lstlisting}[language=C++]
%LocalitySort(&DFList->Df,Cache->HeaderP->GroupCount,sizeof(*DFList));
%\end{lstlisting}
