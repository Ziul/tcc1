\begin{resumo}[Abstract]
 \begin{otherlanguage*}{english}
   
   %This paper presents a study of alternatives for better presentation of search results to repositories and packages managers present in Linux distributions in order to sort the most interesting results to the user based on the values of input and inclusion of consideration of input contains errors spelling, either through ignorance of the name of the package, either by casual misspellings. To aid in the study, we used string matching algorithms for sorting candidates and include the possibility of typos.


This paper presents proposals for better presentation of search results in managers package repositories present in Linux distributions, with the intention of presenting the most relevant results to the user based on the input values and also to treat the cases where the input has spelling mistakes are due to ignorance of the name of the package, either by casual misspellings. Such proposals were constructed from string matching algorithms in the literature and in the Python's libraries.
   \vspace{\onelineskip}
 
   \noindent 
   \textbf{Key-words}: Linux. Packages Managers. APT. Python.
 \end{otherlanguage*}
\end{resumo}
